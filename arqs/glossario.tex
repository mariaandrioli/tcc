% ---
% entradas do glossario
% ---
\newglossaryentry{datawarehouse}{
  name={\uppercase{data warehouse}},
  text={data warehouse},
  description={Sistema de gerenciamento de grandes quantidades de dados usado principalmente na área de Business Intelligence}
}

\newglossaryentry{framework}{
  name={\uppercase{framework}},
  text={framework},
  description={Abstração de na qual um \textit{software} provê uma funcionalidade genérica que pode ser incrementada com código do usuário}
}

\newglossaryentry{benchmark}{
  name={\uppercase{benchmark}},
  text={benchmark},
  description={Execução de um programa várias vezes com objetivo se avaliar sua performance}
}

\newglossaryentry{tuning}{
  name={\uppercase{tuning}},
  text={tuning},
  description={Melhora da performance de um sistema através da mudança de partes do sistema que mais influenciam sua execução}
}

\newglossaryentry{hardware}{
  name={\uppercase{hardware}},
  text={hardware},
  description={Todo componente físico de um computador}
}

\newglossaryentry{software}{
  name={\uppercase{software}},
  text={hardware},
  description={Conjunto de instruções, dados ou programas usados para operar computadores ou executar tarefas específicas em um computador}
}

\newglossaryentry{datacenters}{
  name={\uppercase{datacenters}},
  text={datacenters},
  description={Espaço físico dedicado que reúni sistemas de computadores ou de armazenamento}
}

\newglossaryentry{buffer}{
  name={\uppercase{buffer}},
  text={buffer},
  description={Região de memória computacional usada para armazenar dados temporariamente}
}

\newglossaryentry{byte}{
  name={\uppercase{byte}},
  text={byte},
  description={Unidade de dados que contém 8 \gls{bit}s}
}

\newglossaryentry{bit}{
  name={\uppercase{bit}},
  text={bit},
  description={Menor unidade de informação que pode ser armazenada. Pode possuir valores 0 ou 1}
}

\newglossaryentry{heap}{
  name={\uppercase{heap}},
  text={heap},
  description={Estrutura de dados baseada em árvores}
}

\newglossaryentry{javavirtualmachine}{
  name={\uppercase{java virtual machine}},
  text={Java Virtual Machine},
  description={Máquina virutal que executa programas na linguagem Java ou em outra linguagem de programação caso estejam compilados como código binário Java}
}

\newglossaryentry{mergesort}{
  name={\uppercase{merge sort}},
  text={Merge Sort},
  description={Eficiente algoritmo de ordenação baseado em comparação de valores}
}

\newglossaryentry{businessintelligence}{
  name={\uppercase{Business Intelligence}},
  text={Business Intelligence},
  description={Área da computação que armazena de analisa dados produzidos por uma entidade}
}
% ---