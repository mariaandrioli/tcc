%%%%%%%%%%%%%%%%%%%%%%%%%%%%%%%%%%%%%%%%%%%%%%%%%%%%%%%
% Arquivo para entrada de dados para a parte pré textual
%%%%%%%%%%%%%%%%%%%%%%%%%%%%%%%%%%%%%%%%%%%%%%%%%%%%%%%
% 
% Basta digitar as informações indicidas, no formato 
% apresentado.
%
%%%%%%%
% Os dados solicitados são, na ordem:
%
% tipo do trabalho
% componentes do trabalho 
% título do trabalho
% nome do autor
% local 
% data (ano com 4 dígitos)
% orientador(a)
% coorientador(a)(as)(es)
% arquivo com dados bibliográficos
% instituição
% setor
% programa de pós gradução
% curso
% preambulo
% data defesa
% CDU
% errata
% assinaturas - termo de aprovação
% resumos & palavras chave
% agradecimentos
% dedicatoria
% epígrafe


% Informações de dados para CAPA e FOLHA DE ROSTO
%----------------------------------------------------------------------------- 
\tipotrabalho{Monografia}
%    {Relatório Técnico}
%    {Dissertação}
%    {Tese}
%    {Monografia}
% {Trabalho Acadêmico}

% Marcar Sim para as partes que irão compor o documento pdf
%----------------------------------------------------------------------------- 
 \providecommand{\terCapa}{Sim}
 \providecommand{\terFolhaRosto}{Sim}
 \providecommand{\terTermoAprovacao}{Sim}
 \providecommand{\terDedicatoria}{Nao}
 \providecommand{\terFichaCatalografica}{Nao}
 \providecommand{\terEpigrafe}{Sim}
 \providecommand{\terAgradecimentos}{Sim}
 \providecommand{\terErrata}{Nao}
 \providecommand{\terListaFiguras}{Sim}
 \providecommand{\terListaQuadros}{Sim}
 \providecommand{\terListaTabelas}{Nao}
 \providecommand{\terListaCodigos}{Sim}
 \providecommand{\terSiglasAbrev}{Nao}
 \providecommand{\terResumos}{Sim}
 \providecommand{\terSumario}{Sim}
 \providecommand{\terAnexo}{Nao}
 \providecommand{\terApendice}{Nao}
 \providecommand{\terIndiceR}{Nao}
%----------------------------------------------------------------------------- 

\titulo{Otimização do Hadoop MapReduce através do tuning dos parâmetros de configuração}
\autor{Maria Teresa Kravetz Andrioli}
\local{Curitiba}
\data{2022} %Apenas ano 4 dígitos

% Orientador ou Orientadora
\orientador{Prof. Dr. Luiz Carlos P. Albini}
%Prof Emílio Eiji Kavamura, MSc}
\orientadora{}
% Pode haver apenas uma orientadora ou um orientador
% Se houver os dois prevalece o feminino.

% Em termos de coorientação, podem haver até quatro neste modelo
% Sendo 2 mulhere e 2 homens.
% Coorientador ou Coorientadora
\coorientador{}%Prof Morgan Freeman, DSc}
\coorientadora{}

% Segundo Coorientador ou Segunda Coorientadora
\scoorientador{}
%Prof Jack Nicholson, DEng}
\scoorientadora{}
%Prof\textordfeminine~Ingrid Bergman, DEng}
% ----------------------------------------------------------
\addbibresource{referencias.bib}

% ----------------------------------------------------------
\instituicao{%
Universidade Federal do Paraná}

\def \ImprimirSetor{Setor de Ciências Exatas}%
%Setor de Tecnologia}

\def \ImprimirProgramaPos{}%Programa de Pós Graduação em Engenharia de Construção Civil}

\def \ImprimirCurso{Curso de Ciência da Computação}%
%Curso de Engenharia Civil}

\preambulo{
  Trabalho apresentado como requisito parcial para a obtenção do grau de Bacharel em Ciência da Computação no curso de Ciência da Computação, Setor de Ciências Exatas da Universidade Federal do Paraná}
%do grau de Bacharel em Expressão Gráfica no curso de Expressão Gráfica, Setor de Exatas da Universidade Federal do Paraná}

%----------------------------------------------------------------------------- 

\newcommand{\imprimirCurso}{}
%Programa de P\'os Gradua\c{c}\~ao em Engenharia da Constru\c{c}\~ao Civil}

\newcommand{\imprimirDataDefesa}{Maio de 2022}

\newcommand{\imprimircdu}{
02:141:005.7}

% ----------------------------------------------------------
\newcommand{\imprimirerrata}{
Elemento opcional da \cites[4.2.1.2]{NBR14724:2011}. Exemplo:

\vspace{\onelineskip}

FERRIGNO, C. R. A. \textbf{Tratamento de neoplasias ósseas apendiculares com
reimplantação de enxerto ósseo autólogo autoclavado associado ao plasma
rico em plaquetas}: estudo crítico na cirurgia de preservação de membro em
cães. 2011. 128 f. Tese (Livre-Docência) - Faculdade de Medicina Veterinária e
Zootecnia, Universidade de São Paulo, São Paulo, 2011.

\begin{table}[htb]
\center
\footnotesize
\begin{tabular}{|p{1.4cm}|p{1cm}|p{3cm}|p{3cm}|}
  \hline
   \textbf{Folha} & \textbf{Linha}  & \textbf{Onde se lê}  & \textbf{Leia-se}  \\
    \hline
    1 & 10 & auto-conclavo & autoconclavo\\
   \hline
\end{tabular}
\end{table}}

% Comandos de dados - Data da apresentação
\providecommand{\imprimirdataapresentacaoRotulo}{}
\providecommand{\imprimirdataapresentacao}{}
\newcommand{\dataapresentacao}[2][\dataapresentacaoname]{\renewcommand{\dataapresentacao}{#2}}

% Comandos de dados - Nome do Curso
\providecommand{\imprimirnomedocursoRotulo}{}
\providecommand{\imprimirnomedocurso}{}
\newcommand{\nomedocurso}[2][\nomedocursoname]
  {\renewcommand{\imprimirnomedocursoRotulo}{#1}
\renewcommand{\imprimirnomedocurso}{#2}}


% ----------------------------------------------------------
\newcommand{\AssinaAprovacao}{

\assinatura{%\textbf
   {Professora} \\ UFPR}
   \assinatura{%\textbf
   {Professora}}
   \assinatura{%\textbf
   {Professora}}
   %\assinatura{%\textbf{Professor} \\ Convidado 4}
      
   \begin{center}
    \vspace*{0.5cm}
    %{\large\imprimirlocal}
    %\par
    %{\large\imprimirdata}
    \imprimirlocal, \imprimirDataDefesa.
    \vspace*{1cm}
  \end{center}
  }
  
% ----------------------------------------------------------
%\newcommand{\Errata}{%\color{blue}
%Elemento opcional da \textcite[4.2.1.2]{NBR14724:2011}. Exemplo:
%}

% ----------------------------------------------------------
\newcommand{\EpigrafeTexto}{%\color{blue}
\textit{``Tudo que posso fazer é aceitar o pandemônio. \\
		Encontrar felicidade na insanidade única de estar aqui agora.`` \\
		(\textcite{Pandemonium2019} - Pandemonium, The Good Place)}
}

% ----------------------------------------------------------
\newcommand{\ResumoTexto}{%\color{blue}
Um dos grandes desafios da área de computação sempre foi lidar com o uso, armazenamento e controle de dados. \textit{Big Data} refere-se a grandes aglomerados de dados os quais ferramentas tradicionais não conseguem processar de forma otimizada. Assim, foi desenvolvido pelo Google o paradigma de programação \textit{MapReduce}, com o objetivo de simplificar esse processo. O \textit{Hadoop} é um \textit{\gls{framework}} criado pela Apache para armazenar e processar em paralelo grandes quantidades de dados. Em conjunto, eles formam o \textit{Hadoop MapReduce}, que agrega os benefícios das duas ferramentas a fim de obter uma execução otimizada.

O \textit{Hadoop MapReduce} contém uma grande quantidade de parâmetros de configuração que podem - e devem - ser alterados com base na situação sobre o qual ele está sendo aplicado, o ambiente no qual ele está sendo executado e os objetivos a serem atingidos pela aplicação. A alteração deses parâmetros com objetivo de melhora performance é chamada de \textit{\gls{tuning}} e a boa manipulação desses valores é essencial para que o \textit{\gls{framework}} funcione da melhor maneira possível.

Dessa forma, o presente trabalho apresenta as ferramentas mencionadas, os valores iniciais dos parâmetros e os experimentos realizados e resultados obtidos após o \textit{\gls{tuning}} das configurações do \textit{Hadoop MapReduce}.
} 

\newcommand{\PalavraschaveTexto}{%\color{blue}
Big Data; Hadoop; MapReduce; otimização; parâmetros.}

% ----------------------------------------------------------
\newcommand{\AbstractTexto}{%\color{blue}
One of the biggest challenges of the computing area has always been dealing with data use, storage and manipulation. Big Data is the term that defines big conglomerates of data which cannot be processed by traditional tools in an optimized way. With that in mind, Google has developed the programming paradigm called MapReduce to simplify this process. Hadoop is a frameworkc created by Apache to storage and process big amount of data in parallel. Together, they make Hadoop MapReduce, which aggregates the benfits from both tools in order to obtain an optimized performance.

Hadoop MapReduce has a vast amount of configuration parameters  that can - and should - be altered according to the application, the execution environment and which goals are to achieved after your execution. Altering those parameters in order to have an improvement in performance is called tuning and handling those values well is essential for the framework to work in the best way possible.

Thereby, the present work presents the aforementioned tools, the default values for the parameteres and the experiments and results obtained after tuning Hadoop Map Reduce configuration.
}
% ---
\newcommand{\KeywordsTexto}{%\color{blue}
Big Data; Hadoop; MapReduce; otimization; framework; tuning; parameters.
}

% ----------------------------------------------------------
\newcommand{\Resume}
{%\color{blue}
%Il s'agit d'un résumé en français.
} 
% ---
\newcommand{\Motscles}
{%\color{blue}
 %latex. abntex. publication de textes.
}

% ----------------------------------------------------------
\newcommand{\Resumen}
{%\color{blue}
%Este es el resumen en español.
}
% ---
\newcommand{\Palabrasclave}
{%\color{blue}
%latex. abntex. publicación de textos.
}

% ----------------------------------------------------------
\newcommand{\AgradecimentosTexto}{%\color{blue}
ADICIONAR AGRADECIMENTOS
}

% ----------------------------------------------------------
\newcommand{\DedicatoriaTexto}{%\color{blue}
\textit{ Este trabalho é dedicado às crianças adultas que,\\
   quando pequenas, sonharam em se tornar cientistas.}
	}

