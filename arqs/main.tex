%% abtex2-modelo-trabalho-academico.tex, v-1.9.2 laurocesar
%% Copyright 2012-2017 by abnTeX2 group at http://abntex2.googlecode.com/ 
%%
%% This work may be distributed and/or modified under the
%% conditions of the LaTeX Project Public License, either version 1.3
%% of this license or (at your option) any later version.
%% The latest version of this license is in
%%   http://www.latex-project.org/lppl.txt
%% and version 1.3 or later is part of all distributions of LaTeX
%% version 2005/12/01 or later.
%%
%% This work has the LPPL maintenance status `maintained'.
%% 
%% The Current Maintainer of this work is Emílio Eiji Kavamura,
%% eek.edu@outlook.com; emilio.kavamura@ufpr.br
%% Further information about abnTeX2 are available on 
%%
%% http://abntex2.googlecode.com/
%%
%% https://code.google.com/p/abntex2/issues/ 
%%
%% Further information about UFPR abnTeX2 are available on 
%%
%% https://github.com/eekBR/ufpr-abntex/
%%
%% This work consists of the files 
% 
%          main.tex   programa principal
%      00-dados.tex   entrada de dados 
%    00-pacotes.tex   pacotes carregados no modelo
% 00-pretextual.tex   processamento dos elementos pre-textuais
%          UFPR.sty   ajusta do modelo canonico às normas  UFPR
%
%    referencias.bib
%                     e outras arquivos de imagens
%
%
%------------------------------------------------------------------------
% ------------------------------------------------------------------------
% abnTeX2: Modelo de Trabalho Academico (tese de doutorado, dissertacao de
% mestrado e trabalhos monograficos em geral) em conformidade com 
% ABNT NBR 6023:2018: Informação e documentação - Referências - Elaboração
% ------------------------------------------------------------------------
% ------------------------------------------------------------------------
%
% DATA DE ATUALIZAÇÃO: 2020-06-10

\documentclass[
        % -- opções da classe memoir --
        12pt,                           % tamanho da fonte
        openright,                      % capítulos começam em pág ímpar (insere página vazia caso preciso)
        %twoside,                        % para impressão em verso e anverso. Oposto a oneside
        oneside,
        a4paper,                        % tamanho do papel. 
        % -- opções da classe abntex2 --
        chapter=TITLE,         % títulos de capítulos convertidos em letras maiúsculas
        section=TITLE,         % títulos de seções convertidos em letras maiúsculas
        subsection=Title,      % títulos de subseções convertidos em letras maiúsculas
        %subsubsection=TITLE,  % títulos de subsubseções convertidos em letras maiúsculas
        % -- opções do pacote babel --
        english,                        % idioma adicional para hifenização
        %french,                         % idioma adicional para hifenização
        spanish,                        % idioma adicional para hifenização
        portugues,                      % o último idioma é o principal do documento
        %%%%%%%%%%%%
        %eek: colocação da opção para o sumario ter formatação tradicional
        sumario=tradicional             % título no formato tradicional
        ]{abntex2}


\usepackage{UFPR}
% Pacotes básicos 
% ----------------------------------------------------------
\usepackage{lmodern}			% Usa a fonte Latin Modern			
\usepackage[T1]{fontenc}		% Selecao de codigos de fonte.
\usepackage[utf8]{inputenc}		% Codificacao do documento (conversão automática dos acentos)
\usepackage{lastpage}			% Usado pela Ficha catalográfica
\usepackage{indentfirst}		% Indenta o primeiro parágrafo de cada seção.
\usepackage{color}		    	% Controle das cores
\usepackage{graphicx}			% Inclusão de gráficos
\usepackage{microtype} 			% para melhorias de justificação
\usepackage{ifthen}		    	% para montar condicionais
\usepackage[brazil]{babel}		% para utilizar termos em portugues
\usepackage[final]{pdfpages}    % para incluir páginas de arquivos pdf
\usepackage{lipsum}				% para geração de dummy text
\usepackage{csquotes}

%\usepackage[style=long]{glossaries}
%\usepackage{abntex2glossaries}

\usepackage{algorithm} 
\usepackage{algpseudocode} 
\usepackage{cancel} 		% permite representar o cancelamento de termos em texto ou equacoes	
\usepackage{xcolor} 		% cores extendidas	
\usepackage{smartdiagram}   	% gera diagramas a partir de listas
%\usepackage{float} 		% Para a figura ficar na posição correta	    
\usepackage{textcomp} 		% supporte para fontes da Text Companion 
\usepackage{longtable}		% uso de longtable
\usepackage{amsmath}		% simbolos matematicos
\usepackage{lscape}		% páginas em paisagem
\usepackage{multicol}		% mescla de colunas em tabelas
\usepackage{multirow}		% mescla de linhas em tabelas
\usepackage{newfloat} 		% criação do indice de quadros
%\usepackage{caption} 		% configura legenda 
	%[format=plain]
	%\renewcommand\caption[1]{%
    	%\captionsetup{font=small}	% tamanho da fonte 10pt
    	%,format=hang
 	% \caption{#1}}
	%\captionsetup{width=0.8\textwidth}
\captiondelim{-- }
\captiontitlefont{\small}
\captionnamefont{\small}

\definecolor{dkgreen}{rgb}{0,0.6,0}
\definecolor{gray}{rgb}{0.5,0.5,0.5}
\definecolor{mauve}{rgb}{0.58,0,0.82}

% Pacotes de citações BibLaTeX
% ----------------------------------------------------------
\usepackage[style=abnt,
	backref=true,
	backend=biber,
	citecounter=true,
	backrefstyle=three, 
	url=true,
	maxbibnames=99,
    mincitenames=1,
    maxcitenames=2,
    backref=true,
    hyperref=true,
    firstinits=true,
    uniquename=false,
    uniquelist=false]{biblatex}
    
% Espaçamento entre os itens nas referências (espço de uma linha simples)
% ----------------------------------------------------------
\setlength\bibitemsep{\baselineskip}

% Texto padrão para as referências
% ----------------------------------------------------------
\DefineBibliographyStrings{brazil}{%
	 backrefpage  = {Citado \arabic{citecounter} vez na página},		% originally "cited on page"
	 backrefpages = {Citado \arabic{citecounter} vezes nas páginas},	% originally "cited on pages"
	 urlfrom      = {Dispon\'ivel em},
}

% Ajusta indentação de Referencias no ToC
% ----------------------------------------------------------
\defbibheading{bay}[\bibname]{%
  \chapter*{#1}%
  \markboth{#1}{#1}%
  \addcontentsline{toc}{chapter}
  {\protect\numberline{}\bibname}
}

% Formatando o avançao dos títulos no sumário 
% ----------------------------------------------------------
\makeatletter
	\pretocmd{\chapter}{\addtocontents{toc}{\protect\addvspace{-12\p@}}}{}{}
	\pretocmd{\section}{\addtocontents{toc}{\protect\addvspace{-3\p@}}}{}{}
\makeatother

% Para retirar os símbolos <> da URL  
% ----------------------------------------------------------
\DeclareFieldFormat{illustrated}{\addspace #1\isdot}%
%\DeclareFieldFormat{url}{\bibstring{urlform}\addcolon\addspace<\url{#1}>}%
%\DeclareFieldFormat{url}{\bibstring{urlfrom}\addcolon\addspace<\url{#1}>}%
\DeclareFieldFormat{url}{\bibstring{urlfrom}\addcolon \space\addspace{#1}} 
% remove <> em urls de acordo com abnt-6023:2018	

% Ajustar o espaço para a formatação da data
% ----------------------------------------------------------
\DeclareFieldFormat{urldate}{\bibstring{urlseen}\addcolon\addspace #1}%
\DeclareFieldFormat*{note}{\addspace #1}%

% Para ajustar o tamanho da fonte do número da primeira página do capítulo
% comando utilizado na parte textual 
% ----------------------------------------------------------
\makepagestyle{chapfirst}% Just for the first page of a chapter
\makeoddhead{chapfirst}{}{}{\footnotesize{\thepage}}

%%criar um novo estilo de cabeçalhos e rodapés
\makepagestyle{simplestextual}
  %%cabeçalhos
  \makeevenhead{simplestextual} %%pagina par
     {}{}{\footnotesize \thepage}
     
  \makeoddhead{simplestextual} %%pagina ímpar ou com oneside
     {}{}{\footnotesize \thepage}
  %\makeheadrule{simplestextual}{\textwidth}{\normalrulethickness} %linha
  %% rodapé
  \makeevenfoot{simplestextual}
     {}{}{} %%pagina par
      
  \makeoddfoot{simplestextual} %%pagina ímpar ou com oneside
     {}{}{}
     
% Define a formatação dos capítulos póstextuais numerados
% ----------------------------------------------------------
\newcommand{\poschap}[1]{
	\stepcounter{chapter}
	\markboth{#1}{#1}%
	\pdfbookmark[2]{#1}{#1}
	\addtocontents{toc}{\vspace{-0pt}}
	\addcontentsline{toc}{chapter}{\hspace{14.5mm}\textbf{\appendixname~
	\thechapter~- #1}}
	\chapter*{\appendixname\space\space\thechapter~- \uppercase{#1}}%
	{}
}
\newcommand{\refap}[1]{\hyperref[#1]{Apêndice~\ref{#1}}} 	% Referência apÊndices
% uso do tikz e pgfplots
% ----------------------------------------------------------
%\usetikzlibrary{external}
\usetikzlibrary{arrows,calc,patterns,angles,quotes}
\usepackage{pgfplots}
\pgfplotsset{compat=1.15}

% Define o comando para citação de fontes em elementos gráficos (figuras, imagens,...).
% ----------------------------------------------------------
%  AUTOR(ano)
%
% parâmetro é a bibkey da fonte
  
\newcommand{\citefig}[2]{~\Citeauthor*{#1}\citeyear{#1}}

% Define os operadores matemáticos em portugues
% ----------------------------------------------------------
%

\DeclareMathOperator{\tr}{tr}
\DeclareMathOperator{\sen}{sen}
\DeclareMathOperator{\senh}{senh}
%\DeclareMathOperator{\tag}{tag}
\DeclareMathOperator{\tg}{tg}
\DeclareMathOperator{\tagh}{tagh}
\DeclareMathOperator{\tgh}{tgh}
\DeclareMathOperator{\cossec}{cossec}
%\DeclareMathOperator{\sen}{sen}

% Para fazer a listagem de codigos LaTeX na documentação
% ----------------------------------------------------------
\usepackage{listings}

%Comando para fazer 
%    a citação de documentos não publicados e informais e 
%    colocar as referências nas notas de rodapé
% ----------------------------------------------------------

\newcommand{\citenp}[1]{
\cite{#1}\footnote{\fullcite{#1}}}

\newcommand{\textcitenp}[1]{
	\textcite{#1}\footnote{\fullcite{#1}}}
%%%%%%%%%%%%%%%%%%%%%%%%%%%%%%%%%%%%%%%%%%%%%%%%%%%%%%%
% Arquivo para entrada de dados para a parte pré textual
%%%%%%%%%%%%%%%%%%%%%%%%%%%%%%%%%%%%%%%%%%%%%%%%%%%%%%%
% 
% Basta digitar as informações indicidas, no formato 
% apresentado.
%
%%%%%%%
% Os dados solicitados são, na ordem:
%
% tipo do trabalho
% componentes do trabalho 
% título do trabalho
% nome do autor
% local 
% data (ano com 4 dígitos)
% orientador(a)
% coorientador(a)(as)(es)
% arquivo com dados bibliográficos
% instituição
% setor
% programa de pós gradução
% curso
% preambulo
% data defesa
% CDU
% errata
% assinaturas - termo de aprovação
% resumos & palavras chave
% agradecimentos
% dedicatoria
% epígrafe


% Informações de dados para CAPA e FOLHA DE ROSTO
%----------------------------------------------------------------------------- 
\tipotrabalho{Dissertação}
%    {Relatório Técnico}
%    {Dissertação}
%    {Tese}
%    {Monografia}
% {Trabalho Acadêmico}

% Marcar Sim para as partes que irão compor o documento pdf
%----------------------------------------------------------------------------- 
 \providecommand{\terCapa}{Sim}
 \providecommand{\terFolhaRosto}{Sim}
 \providecommand{\terTermoAprovacao}{Sim}
 \providecommand{\terDedicatoria}{Sim}
 \providecommand{\terFichaCatalografica}{Nao}
 \providecommand{\terEpigrafe}{Nao}
 \providecommand{\terAgradecimentos}{Sim}
 \providecommand{\terErrata}{Nao}
 \providecommand{\terListaFiguras}{Nao}
 \providecommand{\terListaQuadros}{Nao}
 \providecommand{\terListaTabelas}{Nao}
 \providecommand{\terSiglasAbrev}{Nao}
 \providecommand{\terResumos}{Sim}
 \providecommand{\terSumario}{Sim}
 \providecommand{\terAnexo}{Nao}
 \providecommand{\terApendice}{Nao}
 \providecommand{\terIndiceR}{Nao}
%----------------------------------------------------------------------------- 

\titulo{Titulo}
\autor{Maria Teresa Kravetz Andrioli}
\local{Curitiba}
\data{2022} %Apenas ano 4 dígitos

% Orientador ou Orientadora
\orientador{Prof. Dr. Luiz Carlos P. Albini}
%Prof Emílio Eiji Kavamura, MSc}
\orientadora{}
% Pode haver apenas uma orientadora ou um orientador
% Se houver os dois prevalece o feminino.

% Em termos de coorientação, podem haver até quatro neste modelo
% Sendo 2 mulhere e 2 homens.
% Coorientador ou Coorientadora
\coorientador{}%Prof Morgan Freeman, DSc}
\coorientadora{}

% Segundo Coorientador ou Segunda Coorientadora
\scoorientador{}
%Prof Jack Nicholson, DEng}
\scoorientadora{}
%Prof\textordfeminine~Ingrid Bergman, DEng}
% ----------------------------------------------------------
\addbibresource{referencias.bib}

% ----------------------------------------------------------
\instituicao{%
Universidade Federal do Paraná}

\def \ImprimirSetor{Setor de Ciências Exatas}%
%Setor de Tecnologia}

\def \ImprimirProgramaPos{}%Programa de Pós Graduação em Engenharia de Construção Civil}

\def \ImprimirCurso{Curso de Ciência da Computação}%
%Curso de Engenharia Civil}

\preambulo{
  Trabalho apresentado como requisito parcial para a obtenção do grau de Bacharel em Ciência da Computação no curso de Ciência da Computação, Setor de Ciências Exatas da Universidade Federal do Paraná}
%do grau de Bacharel em Expressão Gráfica no curso de Expressão Gráfica, Setor de Exatas da Universidade Federal do Paraná}

%----------------------------------------------------------------------------- 

\newcommand{\imprimirCurso}{}
%Programa de P\'os Gradua\c{c}\~ao em Engenharia da Constru\c{c}\~ao Civil}

\newcommand{\imprimirDataDefesa}{Maio de 2022}

\newcommand{\imprimircdu}{
02:141:005.7}

% ----------------------------------------------------------
\newcommand{\imprimirerrata}{
Elemento opcional da \cites[4.2.1.2]{NBR14724:2011}. Exemplo:

\vspace{\onelineskip}

FERRIGNO, C. R. A. \textbf{Tratamento de neoplasias ósseas apendiculares com
reimplantação de enxerto ósseo autólogo autoclavado associado ao plasma
rico em plaquetas}: estudo crítico na cirurgia de preservação de membro em
cães. 2011. 128 f. Tese (Livre-Docência) - Faculdade de Medicina Veterinária e
Zootecnia, Universidade de São Paulo, São Paulo, 2011.

\begin{table}[htb]
\center
\footnotesize
\begin{tabular}{|p{1.4cm}|p{1cm}|p{3cm}|p{3cm}|}
  \hline
   \textbf{Folha} & \textbf{Linha}  & \textbf{Onde se lê}  & \textbf{Leia-se}  \\
    \hline
    1 & 10 & auto-conclavo & autoconclavo\\
   \hline
\end{tabular}
\end{table}}

% Comandos de dados - Data da apresentação
\providecommand{\imprimirdataapresentacaoRotulo}{}
\providecommand{\imprimirdataapresentacao}{}
\newcommand{\dataapresentacao}[2][\dataapresentacaoname]{\renewcommand{\dataapresentacao}{#2}}

% Comandos de dados - Nome do Curso
\providecommand{\imprimirnomedocursoRotulo}{}
\providecommand{\imprimirnomedocurso}{}
\newcommand{\nomedocurso}[2][\nomedocursoname]
  {\renewcommand{\imprimirnomedocursoRotulo}{#1}
\renewcommand{\imprimirnomedocurso}{#2}}


% ----------------------------------------------------------
\newcommand{\AssinaAprovacao}{

\assinatura{%\textbf
   {Professora} \\ UFPR}
   \assinatura{%\textbf
   {Professora}}
   \assinatura{%\textbf
   {Professora}}
   %\assinatura{%\textbf{Professor} \\ Convidado 4}
      
   \begin{center}
    \vspace*{0.5cm}
    %{\large\imprimirlocal}
    %\par
    %{\large\imprimirdata}
    \imprimirlocal, \imprimirDataDefesa.
    \vspace*{1cm}
  \end{center}
  }
  
% ----------------------------------------------------------
%\newcommand{\Errata}{%\color{blue}
%Elemento opcional da \textcite[4.2.1.2]{NBR14724:2011}. Exemplo:
%}

% ----------------------------------------------------------
\newcommand{\EpigrafeTexto}{%\color{blue}
\textit{``Não vos amoldeis às estruturas deste mundo, \\
		mas transformai-vos pela renovação da mente, \\
		a fim de distinguir qual é a vontade de Deus: \\
		o que é bom, o que Lhe é agradável, o que é perfeito.\\
		(Bíblia Sagrada, Romanos 12, 2)}
}

% ----------------------------------------------------------
\newcommand{\ResumoTexto}{%\color{blue}
O resumo deve ressaltar o  objetivo, o método, os resultados e as conclusões do documento. A ordem e a extensão destes itens dependem do tipo de resumo (informativo ou indicativo) e do tratamento que cada item recebe no documento original. O resumo deve ser precedido da referência do documento, com exceção do resumo inserido no próprio documento. (\ldots) As palavras-chave devem figurar logo abaixo do  resumo, antecedidas da expressão Palavras-chave:, separadas entre si por ponto e finalizadas também por ponto.Ter no máximo 500 palavras!!! As palavras chave são separadas por ponto e vírgula.
} 

\newcommand{\PalavraschaveTexto}{%\color{blue}
latex; abntex; editoração de texto.}

% ----------------------------------------------------------
\newcommand{\AbstractTexto}{%\color{blue}
This is the english abstract.
}
% ---
\newcommand{\KeywordsTexto}{%\color{blue}
latex. abntex. text editoration.
}

% ----------------------------------------------------------
\newcommand{\Resume}
{%\color{blue}
%Il s'agit d'un résumé en français.
} 
% ---
\newcommand{\Motscles}
{%\color{blue}
 %latex. abntex. publication de textes.
}

% ----------------------------------------------------------
\newcommand{\Resumen}
{%\color{blue}
%Este es el resumen en español.
}
% ---
\newcommand{\Palabrasclave}
{%\color{blue}
%latex. abntex. publicación de textos.
}

% ----------------------------------------------------------
\newcommand{\AgradecimentosTexto}{%\color{blue}
Os agradecimentos principais são direcionados à Gerald Weber, Miguel Frasson, Leslie H. Watter, Bruno Parente Lima, Flávio de  Vasconcellos Corrêa, Otavio Real Salvador, Renato Machnievscz\footnote{Os nomes dos integrantes do primeiro
projeto abn\TeX\ foram extraídos de \url{http://codigolivre.org.br/projects/abntex/}} e todos aqueles que contribuíram para que a produção de trabalhos acadêmicos conforme as normas ABNT com \LaTeX\ fosse possível.

Agradecimentos especiais são direcionados ao Centro de Pesquisa em Arquitetura da Informação\footnote{\url{http://www.cpai.unb.br/}} da Universidade de Brasília (CPAI), ao grupo de usuários
\emph{latex-br}\footnote{\url{http://groups.google.com/group/latex-br}} e aos novos voluntários do grupo \emph{\abnTeX}\footnote{\url{http://groups.google.com/group/abntex2} e
\url{http://abntex2.googlecode.com/}}~que contribuíram e que ainda
contribuirão para a evolução do \abnTeX.

Os agradecimentos principais são direcionados à Gerald Weber, Miguel Frasson, Leslie H. Watter, Bruno Parente Lima, Flávio de Vasconcellos Corrêa, Otavio Real Salvador, Renato Machnievscz\footnote{Os nomes dos integrantes do primeiro
projeto abn\TeX\ foram extraídos de \url{http://codigolivre.org.br/projects/abntex/}} e todos aqueles que contribuíram para que a produção de trabalhos acadêmicos conforme as normas ABNT com \LaTeX\ fosse possível.
}

% ----------------------------------------------------------
\newcommand{\DedicatoriaTexto}{%\color{blue}
\textit{ Este trabalho é dedicado às crianças adultas que,\\
   quando pequenas, sonharam em se tornar cientistas.}
	}



% compila o indice
% ----------------------------------------------------------

\makeindex

% ---
% GLOSSARIO
% ---
\makeglossaries

% ---
% entradas do glossario
% ---
\newglossaryentry{datawarehouse}{
  name={\uppercase{data warehouse}},
  text={data warehouse},
  description={Sistema de gerenciamento de grandes quantidades de dados usado principalmente na área de Business Intelligence}
}

\newglossaryentry{framework}{
  name={\uppercase{framework}},
  text={framework},
  description={Abstração na qual um \textit{software} provê uma funcionalidade genérica que pode ser incrementada com código do usuário}
}

\newglossaryentry{benchmark}{
  name={\uppercase{benchmark}},
  text={benchmark},
  description={Execução de um programa várias vezes com objetivo de avaliar sua performance}
}

\newglossaryentry{tuning}{
  name={\uppercase{tuning}},
  text={tuning},
  description={Processo de melhora da performance de um sistema através da mudança de componentes que mais influenciam sua execução}
}

\newglossaryentry{hardware}{
  name={\uppercase{hardware}},
  text={hardware},
  description={Todo componente físico de um computador}
}

\newglossaryentry{software}{
  name={\uppercase{software}},
  text={hardware},
  description={Conjunto de instruções, dados ou programas usados para operar computadores ou executar tarefas específicas em um computador}
}

\newglossaryentry{datacenters}{
  name={\uppercase{datacenters}},
  text={datacenters},
  description={Espaço físico dedicado que reúne sistemas de computadores ou de armazenamento}
}

\newglossaryentry{buffer}{
  name={\uppercase{buffer}},
  text={buffer},
  description={Região de memória computacional usada para armazenar dados temporariamente}
}

\newglossaryentry{byte}{
  name={\uppercase{byte}},
  text={byte},
  description={Unidade de dados que contém 8 \gls{bit}s}
}

\newglossaryentry{bit}{
  name={\uppercase{bit}},
  text={bit},
  description={Menor unidade de informação que pode ser armazenada. Pode possuir valores 0 ou 1}
}

\newglossaryentry{heap}{
  name={\uppercase{heap}},
  text={heap},
  description={Estrutura de dados baseada em árvores}
}

\newglossaryentry{javavirtualmachine}{
  name={\uppercase{java virtual machine}},
  text={Java Virtual Machine},
  description={Máquina virtual que executa programas na linguagem Java ou em outra linguagem de programação se estiverem compilados como código binário Java}
}

\newglossaryentry{mergesort}{
  name={\uppercase{merge sort}},
  text={Merge Sort},
  description={Eficiente algoritmo de ordenação baseado em comparação de valores}
}

\newglossaryentry{businessintelligence}{
  name={\uppercase{Business Intelligence}},
  text={Business Intelligence},
  description={Área da computação que armazena de analisa dados produzidos por uma entidade}
}

\newglossaryentry{cores}{
  name={\uppercase{cores}},
  text={cores},
  description={Unidade de processamento de operações computacionais}
}

\newglossaryentry{kernel}{
  name={\uppercase{kernel}},
  text={kernel},
  description={Programa de computador responsável pelo sistema operacional.}
}
% ---

% ---
% Exemplo de configurações do glossairo
\renewcommand*{\glsseeformat}[3][\seename]{\textit{#1}  
 \glsseelist{#2}}
% ---
      

\lstset{frame=tb,
  language=C,
  aboveskip=25pt,
  belowskip=3mm,
  showstringspaces=false,
  columns=flexible,
  basicstyle={\small\ttfamily},
  numbers=left,
  numberstyle=\tiny\color{gray},
  keywordstyle=\color{blue},
  commentstyle=\color{dkgreen},
  stringstyle=\color{mauve},
  breaklines=true,
  breakatwhitespace=true,
  tabsize=3,
  framesep=5pt,
  xleftmargin=20pt,
  xrightmargin=20pt,
  captionpos=b
}

\renewcommand\lstlistlistingname{LISTA DE CÓDIGOS}

\newcommand*{\noaddvspace}{\renewcommand*{\addvspace}[1]{}}
\addtocontents{lof}{\protect\noaddvspace}

% ----------------------------------------------------------
% Início do documento
% ----------------------------------------------------------
\begin{document}
% ----------------------------------------------------------
% Adequando o uppercase titulo dos elementos nas suas respectivas legendas
% Definicoes que n\~ao funcionaram quando colocados no arquivo de estilos ou de pacotes

\renewcommand{\bibname}{{REFER\^ENCIAS}}
\renewcommand{\tablename}{TABELA }
\renewcommand{\figurename}{FIGURA }
\renewcommand{\figureautorefname}{FIGURA}
\renewcommand{\tableautorefname}{TABELA}
\newcommand{\equationname}{EQUA\c{C}\~AO~}
\renewcommand{\equationautorefname}{EQUA\c{C}\~AO~}
\renewcommand{\lstlistingname}{CÓDIGO}
\providecommand*{\lstlistingautorefname}{CÓDIGO}
\providecommand*{\quadroautorefname}{QUADRO}


% Para ajustar o tamanho da fonte do número da primeira página do capítulo
\aliaspagestyle{chapter}{chapfirst}% customizing chapter pagestyle

% ELEMENTOS PRÉ-TEXTUAIS
\makeoddhead{chapfirst}{}{}{}
% ----------------------------------------------------------
% Capa
% ----------------------------------------------------------
 \ifthenelse{\equal{\terCapa}{Sim}}{
\imprimircapa}{}

% Folha de rosto
% ----------------------------------------------------------
\imprimirfolhaderosto*

% Inserir a ficha bibliografica
% ----------------------------------------------------------
 \ifthenelse{\equal{\terFichaCatalografica}{Sim}}
 {\insereFichaCatalografica{}\cleardoublepage}
 {}

% Inserir errata
% ----------------------------------------------------------
 \ifthenelse{\equal{\terErrata}{Sim}}
 {\begin{errata}%\color{blue}
   \imprimirerrata
  \end{errata}}
 {}

% Inserir folha de aprovação
% ----------------------------------------------------------
\ifthenelse{\equal{\terTermoAprovacao}{Sim}}{
\insereAprovacao}{}

% Dedicatória
% ----------------------------------------------------------
\ifthenelse{\equal{\terDedicatoria}{Sim}}{
\begin{dedicatoria}
   \vspace*{\fill}
   \centering
   \noindent
   \DedicatoriaTexto
   \vspace*{\fill}
\end{dedicatoria}
}{}

% Agradecimentos
% ----------------------------------------------------------

 \ifthenelse{\equal{\terAgradecimentos}{Sim}}
 {\begin{agradecimentos}
    \AgradecimentosTexto
  \end{agradecimentos}
  }{}
% Epígrafe
% ----------------------------------------------------------

\ifthenelse{\equal{\terEpigrafe}{Sim}}{
\begin{epigrafe}
    \vspace*{\fill}
	\begin{flushright}
        \EpigrafeTexto
	\end{flushright}
\end{epigrafe}
}{}

% RESUMOS
% ----------------------------------------------------------
% resumo em português
%\setlength{\absparsep}{18pt} % ajusta o espaçamento dos parágrafos do resumo
 \ifthenelse{\equal{\terResumos}{Sim}}{
\begin{resumo}
    \ResumoTexto
    
    %\vspace{\onelineskip}
    \noindent 
    \textbf{Palavras-chaves}: \PalavraschaveTexto
\end{resumo}

%% resumo em inglês
\begin{resumo}[ABSTRACT]
 \begin{otherlanguage*}{english}
   \AbstractTexto
   
   %\vspace{\onelineskip}
   \noindent 
   \textbf{Key-words}: \KeywordsTexto
 \end{otherlanguage*}
\end{resumo}


% resumo em francês 
\ifthenelse{\equal{\Resume}{}}
{}
{
 \begin{resumo}[RESUME]%Résumé
  \begin{otherlanguage*}{french}
     \Resume
     
     %\vspace{\onelineskip}
     \noindent      
     \textbf{Mots clés}: \Motscles
  \end{otherlanguage*}
 \end{resumo}
} 

% resumo em espanhol
\ifthenelse{\equal{\Resume}{}}{}
{ \begin{resumo}[RESUMEN]
  \begin{otherlanguage*}{spanish}
    \Resumen 
   
   %\vspace{\onelineskip}
   \noindent    
    \textbf{Palabras clave}: \Palabrasclave
  \end{otherlanguage*}
 \end{resumo}
}
}{}

% inserir lista de ilustrações
% ----------------------------------------------------------
\ifthenelse{\equal{\terListaFiguras}{Sim}}{
%\pdfbookmark[0]{\listfigurename}{lof}
\listoffigures*
\cleardoublepage
}{}

% inserir lista de quadros
% ----------------------------------------------------------
\ifthenelse{\equal{\terListaQuadros}{Sim}}{
%\pdfbookmark[0]{\listtablename}{lot}
\listofquadros*
\cleardoublepage
}{}

% inserir lista de tabelas
% ----------------------------------------------------------
\ifthenelse{\equal{\terListaTabelas}{Sim}}{
%\pdfbookmark[0]{\listtablename}{lot}
\listoftables*
\cleardoublepage
}{}


% inserir lista de abreviaturas e siglas 
% inserir lista de símbolos
% ----------------------------------------------------------

 \ifthenelse{\equal{\terSiglasAbrev}{Sim}}{
    \imprimirlistadesiglas
    \cleardoublepage
    \imprimirlistadesimbolos
    \cleardoublepage
 }{}

% inserir o sumario
\makeoddhead{chapfirst}{}{}{}
% ----------------------------------------------------------
\ifthenelse{\equal{\terSumario}{Sim}}{
%\pdfbookmark[0]{\contentsname}{toc}
\tableofcontents*
%\cleardoublepage
}{}
 

 
 


% ----------------------------------------------------------
% ELEMENTOS TEXTUAIS
% ----------------------------------------------------------
\textual % \pagestyle{textualUFPR}

\pagestyle{simplestextual}
% sugerido por Youssef Cherem 20170316
% https://mail.google.com/mail/u/0/?tab=wm#inbox/15ad3fe6f4e5ff1f

% Introdução (exemplo de capítulo sem numeração, mas presente no Sumário)
% ----------------------------------------------------------
\chapter{INTRODUÇÃO} \label{cha:introducao}

\section{CONTEXTO} \label{sec:contexto}

O uso, armazenamento e controle de dados é um tema muito discutido na área de computação desde seus primórdios até os dias de hoje. Dessa forma, muitos métodos e algoritmos e termos surgiram ao longo do tempo com o objetivo de gerenciar de forma eficiente esses dados. O surgimento de tais ferramentas computacionais e métodos de armazenamento é de grande importância para a evolução da área.

Atualmente, os métodos mais comuns são bancos de dados relacionais e \textit{\gls{datawarehouse}s} usando computação em nuvem \cite{PastAndFutureTrendsData19}. Além disso, pesquisas nos campos de mineração de dados e aprendizagem de máquina cresceram bastante recentemente, de modo a prover técnicas que permitissem analisar dados complexos e variados entre si \cite{ProgrammingBigData22}. Um grande desafio é o fato de algoritmos sequenciais não serem otimizados o suficiente para lidar com dados em grande quantidade. Assim, computadores de alta performance, com múltiplos \textit{cores}, sistemas na nuvem e algoritmos paralelos e distribuídos são usados para lidar com esses empecilhos de \textit{Big Data} \cite{ProgrammingBigData22}.

\textit{Big Data} refere-se a grandes conglomerados de dados complexos sobre os quais não é possível aplicar ferramentas tradicionais de processamento, armazenamento ou análise \cite{OptmizationSoftwareHadoop18}. Estima-se que em 2025 os dados atuais criados, capturados ou replicados atinjam 175 Zettabytes, ou seja 175.000.000.000 Gigabytes \cite{DigitalizationWorld18}.

A fim de lidar com essa enorme quantidade de dados, foi desenvolvido pelo Google o \textit{MapReduce}, um modelo de programação com uma implementação associada feito para processar e gerar grandes conglomerados de dados. Esse modelo é inspirado nos conceitos de mapear e reduzir, ou seja, aplicar uma operação que conecta cada item da base de dados a um determinado par de chaves e valores, e então aplicar uma operação de reduzir, que une os valores que compartilham chaves \cite{MapReduce08}. Com essas operações é possível paralelizar dados em grandes quantidades e utilizar mecanismos de reutilização para facilitar a busca e a manipulação destes.

Um dos \textit{\gls{framework}s} mais populares que utiliza o \textit{MapReduce} é o \textit{Hadoop}, desenvolvido pela Apache em 2006 e capaz de armazenar e processar de Gigabytes a Petabytes de dados eficientemente, optando por usar múltiplos computadores (\textit{clusters}) em paralelo \cite{HadoopBook15}.

\section{OBJETIVO} \label{sec:objetivo}

O \textit{Hadoop MapReduce} é um \textit{\gls{framework}} extremamente personalizável e adaptável. Dessa forma, frequentemente utiliza-se o processo de \textit{tuning}, que consiste em modificar os mais de 190 parâmetros desse \textit{\gls{framework}} de modo a maximizar a eficiência de um \textit{cluster Hadoop}. Esses parâmetros podem ser alterados em diversas combinações e podem ter efeitos tanto no \textit{cluster} quantos nas tarefas (\textit{jobs}) do processo.

Esse trabalho tem como objetivo avaliar o comportamento do \textit{Hadoop MapReduce} antes e depois do \textit{tuning} de alguns parâmetros de configuração, observando através de métricas de \textit{benchmark} se houve melhora na performance, considerando medidas como tempo e uso de memória.

\section{ESTRUTURA DO TRABALHO} \label{sec:estrtura}

[TODO: ESTRUTURA DO TRABALHO]
\chapter{REFERENCIAL TEÓRICO} \label{cha:refteorico}

Este capítulo tem como objetivo apresentar detalhadamente os conceitos técnicos que serão utilizados ao longo do trabalho. A \autoref{sec:clusters} introduz o conceito de \textit{clusters}. A \autoref{sec:mapreduce} apresenta o \textit{MapReduce}, o modelo de manipulação de dados feito pelo Google e a \autoref{sec:hadoop} trata do \textit{Hadoop}, o \textit{\gls{framework}} desenvolvido pela Apache. Por fim, a \autoref{sec:virtualizacao} explica virtualização, contêiners e a ferramenta Docker.

\section{CLUSTERS} \label{sec:clusters}

Um \textit{cluster} é um conjunto de computadores que trabalham juntos paralelamente em uma determinada aplicação. Cada computador desse conjunto é usualmente chamado de nodo. Além disso, existem diversas categorias de \textit{clusters}, dependendo do problema que eles buscam computar \cite{GoldmanApache12}.

Algumas aplicações comuns de \textit{clusters} são modelagem de clima, simulação de acidentes automotivos, mineração de dados e aplicações da área de astrofísica. Além disso, é comumente visto em aplicações comerciais como bancos e serviços de email \cite{ClusterGridCloudComparison11}.

Uma das maiores vantagens desse tipo de instalação é a tolerância de falhas, pois os sistemas conseguem continuar suas tarefas caso um nodo pare de funcionar. Além disso, é altamente escalável com a adição de novos nodos, não precisa de manutenção frequente e tem um gerenciamento centralizado. Por fim, uma das suas maiores possíveis vantagens é o balanceamento de carga, que busca atingir o equilíbrio entre as tarefas de cada nodo de modo a otimizar os recursos \cite{ClusterGridCloudComparison11}.

\section{MAPREDUCE} \label{sec:mapreduce}

\textit{MapReduce} é um modelo de programação associado a uma implementação que tem como objetivo processar, manipular e gerar grandes \textit{datasets} de modo eficiente, escalável e com aplicações no mundo real. As computações acontecem de acordo com funções de mapeamento e redução e o sistema do \textit{MapReduce} paraleliza essas computações entre grandes \textit{clusters}, lidando com possíveis falhas, escalonamentos e uso eficiente de rede e discos \cite{MapReduce08}.

As operações de mapeamento e redução são baseadas em conceitos presentes em linguagens funcionais e fazem com que seja possível fazer diversas reutilizações, assim lidando com tolerância de falhas \cite{MapReduce08}.

\subsection{Modelo de programação}\label{ssec:mapreducemodelo}

A computação recebe um conjunto de pares (CHAVE, VALOR) e produz um conjunto de pares de (CHAVE, VALOR). O usuário cria as funções \textit{Map} e \textit{Reduce} de acordo com seu uso. \textit{Map} recebe um único par (CHAVE, VALOR) e produz um conjunto intermediário de pares. Em seguida, a biblioteca \textit{MapReduce} agrupa os valores com a mesma chave, os quais servirão de entrada para a função \textit{Reduce}. Nesse momento a função \textit{Reduce} unifica os valores com a mesma chave de modo a criar um conjunto menor, sendo possível dessa forma lidar com listas muito grandes para a memória disponível \cite{MapReduce08}. 

Entre o momento que são executadas as funções \textit{Map} e \textit{Reduce}, existe a fase \textit{Shuffle}, que é criada automaticamente em tempo de execução e executa operações de ordenação (\textit{sort}) e junção (\textit{merge}) \cite{ProHadoop09}. Para \textcite{HadoopBook15}, a operação \textit{Shuffle} é um dos fatores mais influentes no bom desempenho de aplicações \textit{MapReduce}, uma vez que operações de ordenação e junções podem prejudicar ou melhorar muito um algoritmo conforme sua implementação.

Como exemplo, considere-se o problema de contar quantas vezes determinada palavra aparece em um documento. Nesse problema, as funções \textit{Map} e \textit{Reduce} seriam similares aos seguintes pseudocódigos \cite{MapReduce08}:

\begin{lstlisting}[caption={Exemplo de função Map em pseudocódigo adaptado de \cite{MapReduce08}}, label=code:codigo1]
map(String chave, String valor):
// chave: nome do documento
// valor: conteudo do documento

  para cada palavra W em valor:
    criaIntermediario(W, 1);
\end{lstlisting}

% \newpage
\begin{lstlisting}[caption={Exemplo de função Reduce em pseudocódigo adaptado de \cite{MapReduce08}}, label=code:codigo2]
reduce(String chave, Iterador valores):
// chave: uma palavra
// valores: lista de contagens

  int resultado = 0;
  para cada V em valores:
    resultado = resultado + 1;
  cria(resultado);
\end{lstlisting}

\newpage
A função \textit{Map} gera um objeto intermediário de cada palavra associada a uma lista do seu número de ocorrências no texto e a função \textit{Reduce} soma os valores até que essas ocorrências por palavras sejam totalizadas. Além disso, o usuário cria um configuração de \textit{MapReduce} com os parâmetros de entrada e saída e eventuais parâmetros de \textit{tuning}.

Para exemplificar ainda mais, considere um arquivo de texto com três linhas nas quais estão as seguintes frases, respectivamente, uma em cada linha: "vamos para casa", "na minha casa", "para na casa". Nesse exemplo, a função \textit{Map} é chamada três vezes, uma para cada linha, gerando os pares (CHAVE, VALOR) intermediários, um para cada palavra encontrada no texto, como é mostrado na \autoref{fig:fig1}. Para cada palavra distinta ("vamos", "para", "casa", "na", "minha"), é executada a função \textit{Reduce}, que soma quantas vezes cada uma dessas palavras apareceu no texto e gera um arquivo de saída.

\figura{Execução Genérica do MapReduce}{1.00}{fig/fig1.png}{A autora (2022)}{fig1}{}{}

\subsection{Execução do MapReduce}\label{ssec:execucaomapreduce}

O \textit{MapReduce} funciona usando uma estrutura Cliente/Servidor sobre um \textit{cluster} que, segundo \textcite{MapReduce08}, consiste em primeiramente particionar os dados de entrada em blocos de tamanho já definidos e depois distribuir cópias do programa MapReduce entre cada um desses blocos. Existe uma cópia Master, que é responsável por repartir as tarefas (\textit{tasks}), enquanto as demais cópias - denominadas Workers - recebem da Master as tarefas junto com os arquivos de entrada. Ao finalizar a execução de uma tarefa do tipo \textit{Map}, a cópia Worker responsável repassa a Master os arquivos de saída e esta repassa à um Worker esse arquivo com a tarefa de \textit{Reduce}. Por fim, esse worker executa a redução, lendo os pares intermediários que passaram pela fase \textit{Shuffle} e agrupando as instâncias de mesma chave. Quando todas as tarefas \textit{Map} e \textit{Reduce} forem executadas, o programa é finalizado.

Na \autoref{fig:fig1} foi possível ver como o \textit{MapReduce} funcionaria em pequena escala. Uma das maiores vantagens do \textit{MapReduce} é, no entanto, sua escalabilidade, visto que ele permite uma execução distribuída entre uma grande quantidade de nodos. A \autoref{fig:fig2}  representa uma execução genérica do \textit{MapReduce}, descrita no parágrafo acima.

\figura{Exemplo de execução do MapReduce}{.900}{fig/fig2.png}{Adaptado de \cite{MapReduce08}}{fig2}{}{}

\section{HADOOP} \label{sec:hadoop}

\textit{Hadoop} é um \textit{\gls{framework}} desenvolvido na linguagem Java pela Apache Software Foundation com os seguintes princípios arquiteturais, segundo \textcite{ImprovingNavarro18}:
\begin{itemize}
  \item A possibilidade de escalar o sistema ao adicionar nodos no \textit{cluster}.
  \item Possibilidade de funcionar bem em \textit{hardware} que não necessite ser caro e de luxo.
  \item Tolerância a falhas, com implementações que as identificam e permitem que o sistema funcione independente delas acontecerem.
  \item Fornecimento de serviços para que o usuário foque no problema que deseja resolver.
\end{itemize}

Esse \textit{\gls{framework}} disponibiliza ferramentas para que o usuário possa escrever as funções necessárias em diversas linguagens de programação, conforme a necessidade do programador. O \textit{\gls{framework}} funciona na mesma estrutura de Cliente/Servidor apresentada anteriormente, utilizada pelo \textit{MapReduce}. Além disso, oferece ao programador um sistema paralelo e distribuído (\textit{Hadoop HDFS}), com os recursos ocultos ao usuário, mas capaz de lidar com a comunicação entre as máquinas e quaisquer falhas que possam vir a ocorrer e o escalonamento das tarefas.

Além do \textit{Hadoop Map Reduce} e do \textit{Hadoop HDFS}, existem outros subprojetos do Hadoop que compôem sua estrutura principal: o \textit{Hadoop Common}, que fornece ferramentas comuns aos outros subprojetos  o \textit{Hadoop YARN}, um \textit{\gls{framework}} para escalonamento de tarefas e gerenciamento de recursos em \textit{clusters}.

\subsection{Hadoop Common}\label{ssec:hadoopcommon}

Esse subprojeto contém os utilitários e bibliotecas comuns aos outros subprojetos. Por exemplo, funções de manipulação de arquivos, funções auxiliares de serialização de dados, etc \cite{GoldmanApache12}.

\subsection{Hadoop HDFS}\label{ssec:hadoophdfs}

Segundo \textcite{HDFSDesign20} o \textit{Hadoop HDFS} é um sistema de arquivos distribuídos criado para funcionar em \textit{hardware} facilmente obtido e relativamento barato. Suas características principais são a alta capacidade de lidar com falhas e a possibilidade de ser usado com aplicações que possuem grande quantidades de dados como entrada.

Uma instância HDFS é composta de centenas ou milhares de máquinas, cada uma responsável por armazenar uma parte dos dados do sistema. Dessa forma, a rápida detecção e recuperação de falhas é essencial para sua estrutura. Seu \textit{design} foi pensado em aplicações de processamento de dados em blocos e o tamanho de seus arquivos pode variar entre Giga e Terabytes. Além disso, é adaptado para funcionar em diferentes plataformas e prover interfaces que possibilitam mover a aplicação para perto dos dados, permitindo que qualquer operação computacional aplicada seja muito mais eficiente \cite{HDFSDesign20}.

O \textit{Hadoop HDFS} também possui uma estrutura Cliente/Servidor, em que o \textit{Namenode} - responsável por gerenciar o sistema e regular o acessos aos arquivos - é o nodo Master e os \textit{Datanodes} - responsáveis por gerenciar o armazenamento dos nodos aos quais eles estão conectados - são os nodos Worker. O sistema é implementado usando uma estrutura comum de diretórios na qual é possível criar, mover, renomear e remover arquivos, mas ainda não implementa funções como quota de usuários, permissões de acesso ou \textit{links} simbólicos \cite{HDFSDesign20}.

Uma das características essenciais desse sistema é sua capacidade de lidar com grandes quantidades de dados. Segundo \textcite{HDFSDesign20}, isso é feito através do armazenamento dos arquivos como uma sequência de blocos, cujo tamanho e fator de replicação são configuráveis pelo usuário, mas possuem um valor padrão de 64MB. Periodicamente, o \textit{Namenode} recebe dos \textit{Datanodes} um sinal indicando se o funcionamento está correto. O posicionamento e a quantidade de réplicas de um bloco é crítica na análise da boa performance do HDFS, e é um dos fatores que o diferencia de outros sistemas de arquivos distribuídos. Quando executada de forma otimizada, pode aumentar confiabilidade, disponibilidade e uso de redes do sistema.


\subsection{Hadoop MapReduce} \label{ssec:hadoopmapreduce}

O \textit{Hadoop MapReduce} é um \textit{\gls{framework}} que implementa o modelo de programação \textit{MapReduce} para facilitar a criação de aplicações que são capazes de processar grandes quantidades de dados em paralelo em \textit{clusters} de uma forma confiável e com tolerância a falhas. 

Esse \textit{\gls{framework}} é constituído de um \textit{Job} responsável por dividir os arquivos de entrada em blocos independentes que serão processados pelas tarefas (\textit{tasks}) \textit{Map}, ordenados na fase \textit{Shuffle} e inseridos nas tarefas \textit{Reduce} de forma paralela. Os arquivos de entrada e saída são armazenados no sistema de arquivos e o sistema é responsável pelo escalonamento, agendamento e reexecução de tarefas que tenham falhado \cite{HadoopMapReduce22}.

A estrutura Cliente/Servidor tem como nodo Master o \textit{JobTracker} e como nodos Worker os \textit{TaskTrackers}. Como apresentado anteriormente, o nodo Master designa as tarefas e os nodos Worker as executam. A aplicação do programador fornece o local dos arquivos de entrada e saída, a implementação das funções de \textit{Map} e \textit{Reduce} e outros parâmetros de configuração do \textit{Job}. Então, o cliente \textit{Hadoop} envia o \textit{Job} e o arquivo de configuração para o \textit{JobTracker} que distribui as tarefas e controla o funcionamento desse \textit{Job}.

\subsection{Hadoop YARN} \label{ssec:hadoopyarn}

O \textit{Hadoop YARN} é um subprojeto que tem como objetivo dividir as funcionalidades de gerenciamento de recursos de escalonamento de tarefas em módulos diferentes, tendo então um gerenciador global de recursos e um gerenciador local por aplicação \cite{GoldmanApache12}. 

O gerenciador global trabalha em conjunto com um gerenciador de nodos responsável pelos contêiners e pelo monitoramento de uso de recursos, como CPU, memória, uso de disco e uso de redes, assim como o repasse dessas informações para o gerenciador global \cite{HadoopYarn22}.

O YARN foi adicionado ao \textit{Hadoop} versão 2.0 permitindo a separação das camadas de gerenciamento de recursos que possam ser alocados pela aplicação. Com essa camada independente, ilustrada na \autoref{fig:fig3}, as aplicações \textit{MapReduce} podem ser utilizadas em conjunto com aplicações não \textit{MapReduce}. Além disso, esse formato de implementação possibilita economizar custos com o melhor aproveitamento dos nodos \cite{KobylinskaMartins14}. 

\figura{Nova arquitetura do Hadoop 2.0}{.700}{fig/fig3.png}{\cite{KobylinskaMartins14}}{fig3}{}{}

\newpage
\section{VIRTUALIZAÇÃO} \label{sec:virtualizacao}

Virtualização é o processo de criar um ambiente ou uma versão virtual de algum componente computacional, tal como \textit{hardwares}, dispositivos de armazenamento e recursos de rede. A virtualização permite que haja economia nos custos de \textit{hardware}, melhoria na recuperação em caso de falhas e redução da necessidade de espaço físico para \textit{datacenters} \cite{PortnoyVirtualization12}. 

Uma das técnicas da virtualização é a utilização de contêiners. Contêiners, uma virtualização a nível de sistema, permitem que existam múltiplos espaços do usuário por cima de um determinada kernel de sistema. 

\figura{Arquitetura Docker}{0.850}{fig/fig4.png}{\cite{DockerDocs22}}{fig4}{}{}
\newpage
Docker é uma ferramenta que tem como objetivo automatizar a implantação de aplicações em contêiners, cuja estrutura geral está ilustrada na \autoref{fig:fig4}. Essa ferramenta empilha uma implantação de uma aplicação em cima de um ambiente de execução em um contêiner, ou seja, simula um ambiente virtual de modo que o programador possa trabalhar com sua aplicação em produção de forma extremamente configurável para as suas necessidades. Para isso, o Docker utiliza um recurso de imagem, que se refere aos arquivos de sistemas que determinada aplicação necessita para ser executada. Então, esses arquivos são empilhados entre si e servem como uma receita para construção de um ou de múltiplos contêiners \cite{DockerBook14}.


\chapter{OTIMIZAÇÃO DO MAPREDUCE} \label{cha:otimizacaomapreduce}

Como visto anteriormente, o \textit{MapReduce} é um processo com diversas etapas, e portanto, há muitas permutações possíveis das suas configurações que permitem que sua performance seja melhorada. A melhora da performance através da modificação dos parâmetros de configuração é chamado de \textit{tuning}. Para \textcite{HadoopBook15}, existem algum fatores do \textit{Job} a serem considerados - exibidos no \autoref{qua:quadro1} - com objetivo de obter aumento da performance.

\qquadro{Fatores para \textit{Tuning} do \textit{Job MapReduce}}
{\footnotesize
  \centering
  \begin{tabular}{|p{50mm}|p{100mm}|}\hline
    \textbf{ÁREA A SER OTIMIZADA}     & \textbf{COMO OTIMIZAR}                                                                                                                                                                       \\\hline
    \textbf{Quantidade de mapeadores} & Verificar se é possível diminuir a quantidade de execuções da função \textit{Map} de modo que cada uma seja executada por mais tempo. O tempo médio recomendado na literatura é de 1 minuto. \\\hline
    \textbf{Quantidade de redutores}  & Verificar se mais de um redutor está sendo utilizado. O recomendado é que cada tarefa seja executada em média durante 5 minutos e produza 1 bloco de dados.                                  \\\hline
    \textbf{Uso de combinadores}      & Verificar se é possível utilizar algum combinador de dados de modo que a quantidade de data passada à função \textit{Shuffle} seja menor.                                                    \\\hline
    \textbf{Compressão}               & Usualmente, o tempo de execução de um \textit{Job} é diminuído ao usar compressão de dados.                                                                                                  \\\hline
    \textbf{Ajustes na parte Shuffle} & A parte \textit{Shuffle} do processo possui vários parâmetros de \textit{tuning} de memória que podem ser utilizados para a melhora da performance do \textit{Job}.                          \\\hline
  \end{tabular}}
{Adaptado de \cite{HadoopBook15}}{quadro1}{}{}

\section{PARÂMETROS DO MAPREDUCE} \label{sec:parametrosmapreduce}

O \textit{\gls{tuning}} pode ser realizado através da avaliação e mudança dos parâmetros de configuração do \textit{MapReduce}. Cada parâmetro tem um objetivo específico e pode melhorar uma característica do processo. Algumas variáveis mudam configurações no \textit{Job} e algumas afetam o \textit{cluster} diretamente.

O \textit{Hadoop} foi criado para processar grandes arquivos de entradas e é otimizado para \textit{clusters} em máquinas heterogêneas, ou seja, sistemas que usam mais de um tipo de processador com o objetivo de melhorar a performance. Cada \textit{Job} segue a seguinte sequência de passos: configuração, fase \textit{shuffle/sort} e fase \textit{reduce}. O \textit{Hadoop} é responsável por configurar e gerenciar cada um desses passos \cite{ProHadoop09}.

% \textcite{HadoopBook15} explica o processo de \textit{\gls{tuning}} e explicar os parâmetros específicos para otimização da cada passo, detalhado nas seções a seguir.

% \subsection{Configuração de parâmetros}\label{ssec:configuracaooparametros}

O objetivo principal a ser atingido durante o \textit{\gls{tuning}} é possibilitar que a fase \textit{Shuffle} tenha a maior quantidade de memória disponível, ao mesmo tempo que as fases \textit{Map} e \textit{Reduce} tenham memória suficiente para funcionar propriamente. A quantidade de memória disponibilizada a cada \textit{\gls{javavirtualmachine}} é determinada pelos parâmetros \textit{mapreduce.map.memory.mb} e \textit{mapreduce.reduce.memory.mb} \cite{HadoopBook15}.

A fase \textit{Map} recebe de entrada um arquivo e o parâmetro \textit{dfs.blocksize} é responsável por determinar o tamanho do bloco em \textit{\gls{byte}s} sobre o qual esse arquivo será dividido enquanto o parâmetro \textit{dfs.replication} é responsável por determinar quantos blocos serão criados. Ainda nessa fase, os pares (CHAVE, VALOR) são particionados, e essas partições são ordenadas na fase \textit{Shuffle}. O arquivo criado para cada partição é chamado de \textit{spill}. Para cada tarefa \textit{Reduce} existe um \textit{spill}, que passará por uma ordenação do tipo \textit{\gls{mergesort}}, na segunda etapa da fase \textit{Shuffle}, chamada de \textit{Sort} \cite{ProHadoop09}.

Segundo \textcite{HadoopBook15}, a melhor performance da fase \textit{Map} pode ser obtida através da minimalização da quantidade de \textit{spills}, cujos parâmetros de controles são \textit{mapreduce.task.io.sort.factor} - número máximo de entradas para a função \textit{\gls{mergesort}} - e \textit{mapreduce.task.io.sort.mb} - tamanho do \textit{\gls{buffer}} de memória para a saída da função \textit{Map}. O último é especialmente importante e deve ser aumentado sempre que possível. Quando o \textit{\gls{buffer}} atinge a capacidade percentual determinada pelo parâmetro \textit{mapreduce.map.sort.spill.percent} ocorre um vazamento de memória e os conteúdos restantes do arquivo de saída são colocados no disco (no arquivo chamado de \textit{spill}).

Caso exista mais de uma determinada quantidade de arquivos \textit{spill} (quantidade determinada pela propriedade \textit{mapreduce.map.combine.minspills}), a função combinadora é executada novamente antes de ser criado o arquivo de saída. Uma outra otimização possível é o uso de compressores de dados nos arquivos de saída da fase \textit{Map}, processo facilmente habilitado através dos parâmetros \textit{mapreduce.map.output.compress} - valor booleano que habilita a compressão - e \textit{mapreduce.map.output.compress.codec} - classe que vai realizar a compressão \cite{HadoopBook15}.

A fase \textit{Reduce} é otimizada quando os dados intermediários são armazenados na memória, o que não acontece por padrão, visto que sem a alteração dos parâmetros toda a memória é alocada para a fase \textit{Reduce} em si. As propriedades que podem ser alteradas para atingir esse objetivo são \textit{mapreduce.reduce.merge.inmem.threshold}, \textit{mapreduce.reduce.input.buffer.percent} e \textit{mapreduce.reduce.shuffle.merge.percent}, cujos valores ótimos são 0 e 1.0, respectivamente \cite{HadoopBook15}.

Um dos parâmetros que pode ser otimizado na fase \textit{Reduce} é o \textit{mapreduce.reduce. shuffle.parallelcopies}, que determina a quantidade de tarefas que serão executadas em paralelo pelos \textit{reducers} para copiar os arquivos de saída das tarefas \textit{Map} quando estas são finalizadas e seu valor ótimo depende da quantidade de dados que já passou pela fase \textit{Shuffle} \cite{MRONLINELi14}.

Na parte do processo onde as cópias dos arquivos de saída da fase \textit{Map} são executadas, o tamanho do \textit{\gls{buffer}} é controlado pela propriedade \textit{mapreduce.reduce.shuffle.input.buffer.percent}, que especifica a proporção do \textit{\gls{heap}} que será usada para a finalidade mencionada. Quando esse \textit{\gls{buffer}} atinge um número determinado por \textit{mapreduce.reduce.shuffle.merge.percent} ou \textit{mapreduce.reduce.merge.inmem.threshold}, o restante dos arquivos é alocado no disco \cite{HadoopBook15}.

Para cada gerenciador de nodo, o número de partições do arquivo de saída disponilizados para a fase \textit{Reduce} é determinado pela propriedade \textit{mapreduce.shuffle.max.threads}. O valor padrão de 0 determina que o número máximo de tarefas é o dobro do número de processadores da máquina \cite{ProHadoop09}.

Ainda, as propriedades \textit{mapreduce.output.fileoutputformat.compress}, \textit{mapreduce.out- put.fileoutputformat.compress} e \textit{mapreduce.output.fileoutputformat.compress} determinam, respectivamente, se os arquivos de saída do \textit{Job} serão comprimidos, qual será a classe responsável pela compressão e como essa compressão ocorrerá. Por fim, o parâmetro \textit{io.file.buffer.size} determina o tamanho do \textit{\gls{buffer}} que será usado nas operações de leitura e escrita.

Os quadros a seguir resumem os parâmetros mencionados previamente, assim como outros parâmetros relevantes e os valores padrões de cada um:

\qquadro{Parâmetros de ajuste da quantidade de tarefas \textit{Map}}
{\footnotesize
  \centering
  \begin{tabular}{|p{30mm}|p{50mm}|p{35mm}|}\hline
    \textbf{PARÂMETRO}                             & \textbf{DESCRIÇÃO}                                                                                                                   & \textbf{VALOR PADRÃO} \\\hline
    \textbf{mapreduce.task. io.sort.mb}            & Tamanho em \textit{\gls{byte}s} do \textit{\gls{buffer}} de memória na ordenação na saída da função \textit{Map}.                    & 100                   \\\hline
    \textbf{mapreduce.task.io. sort.factor}        & Número máximo de arquivos para juntar simultaneamente durante a ordenação.                                                           & 10                    \\\hline
    \textbf{mapreduce.map. sort.spill.percent}     & Limite de uso do \textit{\gls{buffer}} de memória que pode ser usado antes que os dados sejam colocados em disco.                    & 0.80                  \\\hline
    \textbf{mapreduce.map. combine.minspills}      & Número mínimo de arquivos de vazamento necessário para o combinador funcionar (se um combinador for especificado).                   & 3                     \\\hline
    \textbf{mapreduce.map. output.compress}        & Define se a saída da função \textit{Map} será comprimida.                                                                            & false                 \\\hline
    \textbf{mapreduce.map. output.compress. codec} & Codificador usado na compressão.                                                                                                     & DefaultCodec          \\\hline
    \textbf{mapreduce.shuffle. max.threads}        & Número de tarefas \textit{Worker} por gerenciador de nodos. Esse parâmetro não funciona por \textit{Job} e sim por \textit{cluster}. & 0                     \\\hline
  \end{tabular}}
{Adaptado de \cite{HadoopDocs321}}{quadro2}{}{}

\qquadro{Parâmetros de ajuste da quantidade de tarefas \textit{Reduce}}
{\footnotesize
  \centering
  \begin{tabular}{|p{40mm}|p{50mm}|p{30mm}|}\hline
    \textbf{PARÂMETRO}                                      & \textbf{DESCRIÇÃO}                                                                                                                                        & \textbf{VALOR PADRÃO} \\\hline
    \textbf{mapreduce.task.io.sort. factor}                 & Número máximo de arquivos para juntar simultaneamente durante a ordenação.                                                                                & 10                    \\\hline
    \textbf{mapreduce.reduce.shuffle. parallelcopies}       & Número de tarefas usadas para copiar saída de funções \textit{Map} para funções \textit{Reduce}.                                                          & 5                     \\\hline
    \textbf{mapreduce.reduce.shuffle. maxfetchfailures}     & Número de vezes que uma tarefa \textit{Reduce} tenta obter arquivo de entrada.                                                                            & 10                    \\\hline
    \textbf{mapreduce.reduce.shuffle. input.buffer.percent} & Porcentagem de tamanho do \textit{\gls{heap}} a ser alocada para a saída da fase \textit{Map}.                                                            & 0.70                  \\\hline
    \textbf{mapreduce.reduce.shuffle. merge.percent}        & Limite porcentual da saída da fase \textit{Map} para iniciar o processo de juntar saídas.                                                                 & 0.66                  \\\hline
    \textbf{mapreduce.reduce.merge. inmem.threshold}        & Quantidade de saídas da função \textit{Map} para a saída da fase \textit{Map}.                                                                            & 1000                  \\\hline
    \textbf{mapreduce.reduce.input. buffer.percent}         & Percentual que determina o tamanho do \textit{\gls{heap}} que será utilizado para armazenar saídas da função \textit{Map} durante a fase \textit{Reduce}. & 0.0                   \\\hline
    \textbf{mapreduce.job.reduce. slowstart.completedmaps}  & Percentual de tarefas \textit{Map} que devem estar completas antes que as tarefas \textit{Reduce} sejam iniciadas.                                        & 0.05                  \\\hline
  \end{tabular}}
{Adaptado de \cite{HadoopDocs321}}{quadro3}{}{}

\qquadro{Parâmetros adicionais de configuração}
{\footnotesize
  \centering
  \begin{tabular}{|p{30mm}|p{50mm}|p{30mm}|}\hline
    \textbf{PARÂMETRO}                                          & \textbf{DESCRIÇÃO}                                                                                      & \textbf{VALOR PADRÃO} \\\hline
    \textbf{dfs.blocksize}                                      & Tamanho do bloco em \textit{\gls{byte}s} sobre o qual arquivo de entrada do \textit{Map} será dividido. & 67108864 \gls{byte}s  \\\hline
    \textbf{dfs.replication}                                    & Quantidade de replicação dos blocos.                                                                    & 3                     \\\hline
    \textbf{mapreduce.map. memory.mb}                           & Memória alocada para cada tarefa \textit{Map}.                                                          & 1024                  \\\hline
    \textbf{mapreduce.reduce. memory.mb}                        & Memória alocada para cada tarefa \textit{Reduce}.                                                       & 1024                  \\\hline
    \textbf{mapreduce.output. fileoutputformat.compress}        & Define se a saída do \textit{Job} será comprimida.                                                      & false                 \\\hline
    \textbf{mapreduce.output. fileoutputformat. compress.codec} & Codificador usado na compressão.                                                                        & DefaultCodec          \\\hline
    \textbf{mapreduce.output. fileoutputformat. compress.type}  & Define como os arquivos de saída do \textit{Job} serão comprimidos.                                     & RECORD                \\\hline
    \textbf{io.file.buffer.size}                                & Tamanho do \textit{buffer} que será usado durante operações de leitura e escrita.                       & 4096                  \\\hline
  \end{tabular}}
{Adaptado de \cite{HadoopDocs321}}{quadro4}{}{}



\section{TERASORT} \label{sec:terasort}

A técnica de \textit{benchmarking} consiste na execução, por diversas vezes, de um programa (no caso do \textit{MapReduce}, de um \textit{Job}), a fim de testar se os resultados obtidos são os esperados. Esse processo é eficiente pois é possível comparar os diversos resultados obtidos e obter avaliações de performance \cite{HadoopBook15}.

O \textit{Hadoop} possui várias métricas de \textit{benchmarks} imbutidas que podem ser utilizadas para melhorar o funcionamento do \textit{Job}, cada uma delas com o objetivo de monitorar um fator diferente, como por exemplo, TestDFSIO  - responsável por testar a performance dos dispositivos de entrada e saída - e MRBench/NNBench - testam em conjunto vários \textit{Jobs} pequenos múltiplas vezes \cite{HadoopBook15}.

Nesse trabalho será utilizado a ferramenta \textit{TeraSort}, que ordena o arquivo de entrada completamente. Para \textcite{HadoopBook15}, ela é extremamente eficaz no \textit{benchmarking} do \textit{HDFS} e \textit{MapReduce} em conjunto, já que todo os dados de entrada são processados. Essa ferramenta funciona em três etapas:

\begin{itemize}
  \item \textit{TeraGen} executa um \textit{Job} só de funções \textit{Map} que cria um conjunto de dados binários aleatórios. A execução desse comando é exemplificada no \autoref{code:codigo3}.
  
  \begin{lstlisting}[caption={Exemplo de execução do \textit{TeraGen} adaptado de \cite{HadoopBook15}}, label=code:codigo3]
  hadoop jar hadoop-mapreduce-examples-*.jar \
  teragen <numero de linhas de 100 bytes cada> <diretorio de saida>
  \end{lstlisting}
  
  \item \textit{TeraSort} executa a ordenação dos dados. É nesse passo que é avaliada a performance do \textit{MapReduce}, pois é aqui que as operações do paradigma em si são executadas. A execução desse comando é exemplificada no \autoref{code:codigo4}.
  \begin{lstlisting}[caption={Exemplo de execução do \textit{TeraSort} adaptado de \cite{HadoopBook15}}, label=code:codigo4]
  hadoop jar hadoop-mapreduce-examples-*.jar \
  terasort <diretorio de entrada> <diretorio de saida>
  \end{lstlisting}  

  \item \textit{TeraValidate} executa checagens nos dados ordenados resultantes da fase anterior para verificar se a ordenação foi feita corretamente. A execução desse comando é exemplificada no \autoref{code:codigo5}.
  \begin{lstlisting}[caption={Exemplo de execução do \textit{TeraValidate} adaptado de \cite{HadoopBook15}}, label=code:codigo5]
  hadoop jar hadoop-mapreduce-examples-*.jar \
  teravalidate <arquivo de entrada (diretorio de saida do terasort)> <diretorio de saida>
  \end{lstlisting}

\end{itemize}

Os parâmetros de configuração do \textit{tuning} podem ser usados no comando que executa o \textit{TeraSort}, processo exemplificado na \autoref{fig:fig5}.

\figura{Execução do \textit{TeraSort} com parâmetros de \textit{tuning}}{.900}{fig/fig5.png}{A autora (2022)}{fig5}{}{}
\section{AMBIENTE EXPERIMENTAL} \label{sec:ambienteexperimental}

O ambiente no qual serão realizadas as execuções e testes tem as seguintes configurações: LENOVO Ideapad 310 com processador Intel(R) Core(TM) i5-6200U, 2.30GHz de velocidade de processamento,  memória RAM de 8GB, armazenamento de disco de 1TB e SSD Kingston A400 de 480, com leitura de 500 MB/s e gravação de 450 MB/s. 

No entanto, a imagem Docker é executada no Windows WSL2, Substema do Windows para Linux, que possibilitam aos programadores executar um ambiente Linux mesmo usando um sistema Windows \cite{MicrosoftWSL22}. Nesse ambiente virtual, está sendo executado Linux na distribuição Ubuntu 20.04, Docker na versão 20.10.12 e Hadoop na versão 3.2.1.

Usando a imagem do \textit{Hadoop} para Docker disponibilizada pelo Big Data Europe \cite{BigDataHadoopGithub}, foi configurado um contêiner Docker com um \textit{cluster Hadoop} com 3 \textit{datanodes (workers)}, um \textit{namenode HDFS (master)}, um gerenciador de recursos YARN, um servidor com histórico de operações e um gerenciador de nodos.

\section{RESULTADOS} \label{sec:resultados}

Para que o objetivo de melhora de performance através do \textit{\gls{tuning}} dos parâmetros fosse atingido, foram feitos testes repetidos utilizando o \textit{Terasort} com variação dos valores padrões de cada uma das propriedades relevantes mencionadas previamente de acordo com sugestões de \textcite{HadoopBook15} e \textcite{ProHadoop09}, assim como a análise de cada uma das execuções da ferramenta de \textit{\gls{benchmark}ing} para que fossem decididas quais mudanças teriam mais impacto no resultado. As execuções do \textit{TeraSort} foram realizadas após a geração de dados aleatórios de entrada pelo \textit{TeraGen} de um arquivo de 10GB.

Os primeiros ajustes realizados foram nas propriedades relacionadas à quantidade de memória disponível, isto é, na memória disponibilizada aos \textit{\gls{buffer}s} utilizados nas operações de leitura e escrita e na saída da função \textit{Map}, assim como às funções \textit{Map} e \textit{Reduce}, o que se provou de grande importância porque diminiui consideralmente o tempo de execução do programa. Isso ocorre devido ao fato de que, ao disponibilizar mais memória aos programas, menos dados são copiados para o disco, economizando tempo.

Depois, foram feitas alterações nas propriedades referentes a compressão de arquivos, habilitando-se a compressão na saída da fase \textit{Map} e do \textit{Job} utilizando a classe \textit{org.apache.hadoop.io.compress.Lz4Codec} para realizar essa operação. A compressão é relevante porque permite que os recursos computacionais não fiquem parados esperando operações de entrada e saída em disco finalizarem por estarem trabalhando com arquivos muito grandes.

Os parâmetros relativos aos blocos sobre os quais os arquivos de entrada da função \textit{Map} são separados também foram modificados, tendo seu tamanho aumentado e sua taxa de replicação diminuída. Esses dois fatores permitem que os blocos sejam de maior tamanho, mas não sejam duplicados em nodos diferentes, resultando em menos operações de leitura e escrita e menos uso de rede.

Outras propriedades da tarefa \textit{Map} que foram alteradas são as que determinam a quantidade de arquivos para juntar simultaneamente e a da taxa limite do \textit{\gls{buffer}} para que os valores deste sejem transferidos para o disco.

Em relação à fase \textit{Shuffle}, o único parâmetro modificado foi o \textit{mapreduce.shuffle.max. threads}, determinando-se que apenas uma tarefa \textit{Worker} fosse usada pelo gerenciador de nodos.

Por fim, na fase \textit{Reduce}, as propriedades que tiveram efeito relevante na performance foram as que determinam a quantidade de transferências paralelas executadas pela função \textit{Reduce} durante a fase de cópia dos arquivos da fase \textit{Shuffle}, o limite de arquivos para o processo de junção antes de acontecem transferência para o disco, a quantidade de memória a ser alocada para armazenar arquivos de saída da fase \textit{Map} durante a fase \textit{Shuffle} e a porcentagem de tarefas \textit{Map} que deveriam estar finalizar antes de ser iniciado o processo de \textit{Reduce}.

Os parâmetros e seus valores que foram alterados por afetarem a performance estão descritos no \autoref{qua:quadro5}.

\qquadro{Parâmetros ajustados durante o \textit{tuning}}
{\footnotesize
  \centering
  \begin{tabular}{|p{80mm}|p{25mm}|}\hline
    \textbf{PARÂMETRO}                                        & \textbf{VALOR FINAL} \\\hline
    \textbf{mapreduce.map.output.compress}                    & true                 \\\hline
    \textbf{mapreduce.map.output.compress.codec}              & Lz4Codec             \\\hline
    \textbf{dfs.blocksize}                                    & 335544320            \\\hline
    \textbf{dfs.replication}                                  & 1                    \\\hline
    \textbf{mapreduce.output.fileoutputformat.compress}       & true                 \\\hline
    \textbf{mapreduce.output.fileoutputformat.compress.codec} & Lz4Codec             \\\hline
    \textbf{mapreduce.output.fileoutputformat.compress.type}  & BLOCK                \\\hline
    \textbf{mapreduce.map.memory.mb}                          & 2048                 \\\hline
    \textbf{io.file.buffer.size}                              & 131072               \\\hline
    \textbf{mapreduce.task.io.sort.mb}                        & Lz4Codec             \\\hline
    \textbf{mapreduce.task.io.sort.factor}                    & 256                  \\\hline
    \textbf{mapreduce.map.sort.spill.percent}                 & 400                  \\\hline
    \textbf{mapreduce.shuffle.max.threads}                    & 1.0                  \\\hline
    \textbf{mapreduce.reduce.shuffle.parallelcopies}          & 20                   \\\hline
    \textbf{mapreduce.reduce.merge.inmem.threshold}           & 2000                 \\\hline
    \textbf{mapreduce.reduce.input.buffer.percent}            & 0.8                  \\\hline
    \textbf{mapreduce.job.reduce.slowstart.completedmaps}     & 0.7                  \\\hline
  \end{tabular}}
{A autora(2022)}{quadro5}{}{}

Alguns parâmetros mencionados anteriormente - resumidos no \autoref{qua:quadro6} não tiveram efeito na performance no programa, seja pelo tamanho do arquivo de dados inicial pelas configurações do ambiente experimental. Por causa disso, seus valores padrões foram mantidos nas execuções.

\qquadro{Parâmetros não ajustados durante o \textit{tuning}}
{\footnotesize
  \centering
  \begin{tabular}{|p{75mm}|p{25mm}|}\hline
    \textbf{PARÂMETRO}                                     & \textbf{VALOR FINAL} \\\hline
    \textbf{mapreduce.map.combine.minspills}               & 3                    \\\hline
    \textbf{mapreduce.reduce.shuffle.maxfetchfailures}     & 10                   \\\hline
    \textbf{mapreduce.reduce.shuffle.input.buffer.percent} & 0.7                  \\\hline
    \textbf{mapreduce.reduce.shuffle.merge.percent}        & 0.66                 \\\hline
  \end{tabular}}
{A autora(2022)}{quadro6}{}{}

Os resultados relevantes da aplicação do \textit{\gls{tuning}} são os tempos obtidos na execução do \textit{TeraSort}, os quais podem ser obtidos pela sua saída que mostra, em milisegundos, o tempo utilizado pelas fases \textit{Map} e \textit{Reduce} do processo. Segundo \textcite{Fleming86}, no caso de \textit{\gls{benchmark}ing} de métricas de tempo, é apropriado usar uma média aritmética padrão para avaliar os resultados. Além disso, apresentar os valores mínimos e máximos obtidos na execuções de modo que uma visão geral da performance possa ser exemplificada.

Para geração dessas métricas, o \textit{TeraSort} foi executado 10 vezes com os parâmetros com valores padrão e 10 vezes com os parâmetros com valores alterados.


\figura{Resultados do \textit{tuning}}{.900}{fig/fig6.png}{A autora (2022)}{fig6}{}{}
\newpage
Dessa forma, a \autoref{fig:fig6} acima ilustra os resultados obtidos nesse experimentos antes e depois do processo de aplicação do \textit{\gls{tuning}}. Como é possível ver, a mudança dos parâmetros teve um grande impacto da performance do \textit{Hadoop MapReduce}, diminuindo a execução das suas principais funções em quase 50\%. 

\chapter{CONCLUSÃO} \label{cha:conclusao}

Nesse trabalho foram apresentados o paradigma \textit{MapReduce}, assim como o \textit{\gls{framework}} \textit{Hadoop} e a junção destes, o \textit{Hadoop MapReduce}, assim como o detalhamento do seu funcionamento, sua estrutura e, principalmente, os parâmetros de configuração que se mostraram extremamente influentes na sua boa performance.

A partir do estudo dessas ferramentas assim como testes práticos executados, ficou claro como o bom conhecimento do problema a ser resolvido pela aplicação, a ferramenta que está sendo utilizada e o sistema sobre o qual essa ferramenta está sendo executada são relevantes e devem ser consideradas para a configuração do \textit{Hadoop MapReduce} de forma a obter a melhor performance possível. 

Os experimentos realizados foram aplicados com uma quantidade pequena de dados - em relação a quantidades observadas no mundo real - mas mesmo assim foi possível otimizar os resultados dos \textit{\gls{benchmark}s}. 

Em relação a trabalhos futuros sobre o temo, é interessante ressaltar que já existem em progresso estudos e artigos sobre ferramentas que realizam o \textit{tuning} automático do \textit{Hadoop MapReduce}, inclusive com aprendizagem de máquina e alterações em tempo real. Com isso, temas interessante a serem considerados incluem a análise dessas ferramentas, a comparação entre elas da qual obtém o melhor resultado ou até mesmo a implementação de uma nova ferramenta com técnicas ainda não utilizadas.

% PARTE DA PREPARAÇÃO DA PESQUISA
% ----------------------------------------------------------
%\part{Preparação da pesquisa}
%\input{cap02}
%

% PARTE DOS REFERENCIAIS TEÓRICOS
% ----------------------------------------------------------
%\part{Referenciais teóricos}
%\input{cap03}

% PARTE DOS RESULTADOS
% ----------------------------------------------------------
%\part{Resultados}
%\input{cap05}

% Finaliza a parte no bookmark do PDF
% para que se inicie o bookmark na raiz
% e adiciona espaço de parte no Sumário
% ----------------------------------------------------------
%\phantompart

% ---
% Conclusão (outro exemplo de capítulo sem numeração e presente no sumário)
% ---
% \chapter*[GLOSSÁRIO]{GLOSSÁRIO}
% \addcontentsline{toc}{chapter}{GLOSSÁRIO}
% ---
%\input{cap06}

% ELEMENTOS PÓS-TEXTUAIS
% ----------------------------------------------------------
\postextual

% Ajuste vertical do titulo de referencias no sumário
% ----------------------------------------------------------
\addtocontents{toc}{\vspace{-24pt}}

% Referências bibliográficas
% ----------------------------------------------------------
%\bibliography{referencias}


\begingroup

\printbibliography[heading=bay, notkeyword= {consulta}, notkeyword={npub-informal}]

\endgroup

% ----------------------------------------------------------
% Glossário
% ----------------------------------------------------------

% ---
% Define nome e preâmbulo do glossário
% ---
% \phantompart
\renewcommand{\glossaryname}{GLOSSÁRIO}

% ---
% Estilo de glossário
% ---
\setglossarystyle{tree}

% ---
% Imprime o glossário
% ---
\cleardoublepage
\phantomsection
% \chapter*[GLOSSÁRIO]{GLOSSÁRIO}
\addtocontents{toc}{\vspace{-12pt}}
\addcontentsline{toc}{chapter}{\protect\numberline{}\glossaryname}
\printglossaries
% ---

% ----------------------------------------------------------

% Ajuste vertical dos titulos dos capitulos postextuais
% ----------------------------------------------------------
\addtocontents{toc}{\vspace{-12pt}}

% Glossário
% ----------------------------------------------------------
% Consulte o manual da classe abntex2 para orientações sobre o glossário.
%
% \glossary

% Apêndices
% ----------------------------------------------------------
\ifthenelse{\equal{\terApendice}{Sim}}
{\begin{apendicesenv}

                % Numeração arábica para os apêndices
                % --------------------------------------------------
                \renewcommand{\thechapter}{\arabic{chapter}}
                % Imprime uma página indicando o início dos apêndices
                % \partapendices

                % Existem várias formas de se colocar anexos.
                % O exemplo abaixo coloca 2 apêndices denominados de 
                % DESENVOLVIMENTO DETALHADO DA PINTURA e 
                % ESCOLHA DO MATERIAL DE IMPRESSÃO:
                % ---
                % --- insere um capítulo que é tratado como um apêndice
                %\chapter{DESENVOLVIMENTO DETALHADO DA PINTURA}
                % 
                %\lipsum[29] % gera um parágrafo
                %
                % --- insere um capítulo que é tratado como um apêndice
                %\chapter{ESCOLHA DO MATERIAL DE IMPRESSÃO}
                % 
                %\lipsum[30] % gera um parágrafo

                % --- Insere o texto do arquivo ap01.tex
                % 
                % --- O conteúdo do arquivo pode ser vários anexos ou um único apêndices.
                %     A vantagem de se utilizar este procedimento é de suprimi-lo
                %     das compilações enquanto se processa o resto do documento.

                % --- insere um capítulo que é tratado como um apêndice
\label{ap:ap01}
\poschap{ESCOLHA DO MATERIAL}

\lipsum[30] % gera um parágrafo
\section*{Testes se\c{C}\~aO}

\lipsum[22] % gera um parágrafo

        \end{apendicesenv}
}{}


% Anexos
% ----------------------------------------------------------
\ifthenelse{\equal{\terAnexo}{Sim}}{
        \begin{anexosenv}

                % Numeração arábica para os apêndices
                % --------------------------------------------------
                \renewcommand{\thechapter}{\arabic{chapter}}
                % --- Imprime uma página indicando o início dos anexos
                % \partanexos

                % Existem várias formas de se colocar anexos.
                % O exemplo abaixo coloca 2 anexos denominados de 
                % TABELA DE VALORES e GRÁFICOS DE BALANCEMANTO:
                % ---
                % --- insere um capítulo que é tratado como um anexo
                %\chapter{TABELAS DE VALORES}
                % 
                %\lipsum[31] % gera um parágrafo
                %
                % --- insere um capítulo que é tratado como um anexo
                %\chapter{GRÁFICOS DE BALANCEAMENTO}
                % 
                %\lipsum[32] % gera um parágrafo

                % --- Insere o texto do arquivo ax01.tex
                % 
                % --- O conteúdo do arquivo pode ser vários anexos ou um único anexo.
                %     A vantagem de se utilizar este procedimento é de suprimi-lo
                %     das compilações enquanto se processa o resto do documento.

                % --- insere um capítulo que é tratado como um apêndice
\poschap{anexando ESCOLHA DO MATERIAL}
   
\lipsum[30] % gera um parágrafo
   \section*{anexando testes secao}
        \end{anexosenv}
}{}

% INDICE REMISSIVO
%---------------------------------------------------------------------
\ifthenelse{\equal{\terIndiceR}{Sim}}{
        \phantompart
        \printindex
}{}

\end{document}
