\chapter{CONCLUSÃO} \label{cha:conclusao}

Nesse trabalho foram apresentados o paradigma \textit{MapReduce}, o \textit{\gls{framework}} \textit{Hadoop} e a junção destes, o \textit{Hadoop MapReduce}, assim como o detalhamento do seu funcionamento, estrutura e, principalmente, os parâmetros de configuração que se mostraram extremamente influentes na sua boa performance.

A partir dos estudo e testes práticos executados, percebe-se que o conhecimento do problema a ser resolvido pela aplicação, a ferramenta que está sendo utilizada e o sistema no qual a execução é aplicada são relevantes e devem ser consideradas para a configuração do \textit{Hadoop MapReduce} de forma a obter a melhor performance possível. 

Os experimentos realizados foram aplicados com uma quantidade pequena de dados (em relação a quantidades observadas no mundo real) mas mesmo assim foi possível otimizar os resultados dos \textit{\gls{benchmark}s}. 

Em relação a trabalhos futuros sobre o tema, ressalta-se que já existem em progresso estudos e artigos sobre ferramentas que realizam o \textit{tuning} automático do \textit{Hadoop MapReduce}, inclusive com aprendizagem de máquina e alterações em tempo real. Com isso, temas relevantes a serem considerados incluem a análise dessas ferramentas, a comparação entre  a performance deelas ou até mesmo a implementação de uma nova ferramenta com técnicas ainda não utilizadas.