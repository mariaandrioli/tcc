\section{PARÂMETROS DO \textit{MAPREDUCE}} \label{sec:parametrosmapreduce}

Como mencionado, o \textit{tuning} pode ser realizado através da avaliação e mudança dos parâmetros de configuração do \textit{MapReduce}. Cada parâmetro tem um objetivo específico e pode melhorar uma característica do processo. Algumas variáveis mudam configurações no \textit{Job} e algumas afetam o \textit{cluster} diretamente.

\textit{Hadoop} foi feito para processar grandes arquivos de entradas e é otimizado para \textit{clusters} em máquinas heterogêneas, ou seja, sistemas que usam mais de um tipo de processador com o objetivo de melhorar a performance. Cada \textit{Job} segue a seguinte sequência de passos: configuração, fase \textit{shuffle/sort} e fase \textit{reduce}. O \textit{Hadoop} é responsável por configurar e gerenciar cada um desses passos \cite{ProHadoop09}.

Uma das seções do capítulo 7 de \textcite{HadoopBook15} é dedicado ao processo de \textit{tuning} e explicar os parâmetros específicos para otimização da cada passo, que são explicados a seguir.

\subsection{Configuração de parâmetros}\label{ssec:configuracaooparametros}

Com o propósito de atingir os objetivos demonstrados no \autoref{qua:quadro1}, \textcite{HadoopBook15} sugere configurar os seguintes parâmetros, descritos no \autoref{qua:quadro2} e \autoref{qua:quadro3}:

\qquadro{Parâmetros de ajuste da quantidade de tarefas \textit{Map}}
{\footnotesize
  \centering
  \begin{tabular}{|p{30mm}|p{50mm}|p{35mm}|}\hline
    \textbf{PARÂMETRO}                         & \textbf{DESCRIÇÃO}                                                                                                                   & \textbf{VALOR PADRÃO} \\\hline
    \textbf{mapreduce.task. io.sort.mb}        & Tamanho em \textit{bytes} usado no \textit{buffer} de memória na ordenação na saída da função \textit{Map}.                          & 100                   \\\hline
    \textbf{mapreduce.map. sort.spill.percent} & Limite de uso do \textit{buffer} de memória e de início do processo de vazamento em disco.                                           & 0.80                  \\\hline
    \textbf{mapreduce.task. io.sort.factor}    & Número máximo de entradas para a função de junção na ordenação. \textit{bytes}                                                       & 10                    \\\hline
    \textbf{mapreduce.map. combine.minspills}  & Número mínimo de arquivos de vazamento necessário para o combinador funcionar (se um combinador for especificado)                    & 3                     \\\hline
    \textbf{mapreduce.map. output.compress}    & Define se a saída da função \textit{Map} será comprimida.                                                                            & false                 \\\hline
    \textbf{mapreduce.map. output.compress}    & Codificador usado na compressão.                                                                                                     & DefaultCodec (classe) \\\hline
    \textbf{mapreduce.shuffle. max.threads}    & Número de tarefas \textit{Worker} por gerenciador de nodos. Esse parâmetro não funciona por \textit{Job} e sim por \textit{cluster}. & 0                     \\\hline
  \end{tabular}}
{Adaptado de \cite{HadoopBook15}}{quadro2}{}{}

\qquadro{Parâmetros de ajuste da quantidade de tarefas \textit{Map}}
{\footnotesize
  \centering
  \begin{tabular}{|p{40mm}|p{50mm}|p{30mm}|}\hline
    \textbf{PARÂMETRO}                                       & \textbf{DESCRIÇÃO}                                                                                                                                                                      & \textbf{VALOR PADRÃO} \\\hline
    \textbf{mapreduce.reduce.shuffle. parallelcopies}        & Número de tarefas usadas para copiar saída de funções \textit{Map} para funções \textit{Reduce}.                                                                                        & 5                     \\\hline
    \textbf{mapreduce.reduce.shuffle. maxfetchfailures}      & Número de vezes que uma tarefa \textit{Reduce} tenta obter arquivo de entrada.                                                                                                          & 10                    \\\hline
    \textbf{mapreduce.task.io.sort. factor}                  & Número máximo de arquivos para juntar simultaneamente durante a ordenação.                                                                                                              & 10                    \\\hline
    \textbf{mapreduce.reduce.shuffle. input.buffer.percent} & Porcentagem de tamanho do \textit{heap} a ser alocada para a saída da fase \textit{Map}.                                                                                                & 0.70                  \\\hline
    \textbf{mapreduce.reduce.shuffle. merge.percent}         & Limite porcentual da saída da fase \textit{Map} para iniciar o processo de juntar saídas.                                                                                               & 0.66                  \\\hline
    \textbf{mapreduce.reduce.merge. inmem.threshold}         & Quantidade de saídas da função \textit{Map} para a saída da fase \textit{Map}. Se for igual ou menor a zero, esse fator é definido apenas pelo mapreduce.reduce. shuffle.merge.percent. & 1000                  \\\hline
    \textbf{mapreduce.reduce.input. buffer.percent}          & Percentual que determinar o tamanho do \textit{heap} que será utilizado para armazenar saídas da função \textit{Map} durante a fase \textit{Reduce}.                                    & 0.0                   \\\hline
  \end{tabular}}
{Adaptado de \cite{HadoopBook15}}{quadro3}{}{}

% \subsection{Shuffle e sort}\label{ssec:shufflesort}
% O JobTracker tem um número determiado de lugares para a execução das tarefas Map. Cada divisão feita no split é alocada um desses lugares para ser executada. Manda tarefas para um lugar local da máquina melhora a performance do dispositivo de entrada e saída. Se existem lugares sobrando e o map speculative execution está habilitado, várias instância da uma tarefa map podem ser agendadas para serem executadas. Se ele valor nao estiver habilitado, apenas uma instância de tarefa map será executada por vez.