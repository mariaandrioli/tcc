\chapter{OTIMIZAÇÃO DO MAPREDUCE} \label{cha:otimizacaomapreduce}

Como visto anteriormente, o \textit{MapReduce} é um processo com diversas etapas e, portanto, há muitas permutações possíveis das suas configurações que permitem a melhora da performance que, através da modificação dos parâmetros de configuração, é chamada de \textit{tuning}. Para \textcite{HadoopBook15}, os fatores do \textit{Job} mais relevantes a serem considerados com objetivo de obter aumento da performance são exibidos no \autoref{qua:quadro1}.

\qquadro{Fatores para \textit{Tuning} do \textit{Job MapReduce}}
{\footnotesize
  \centering
  \begin{tabular}{|p{50mm}|p{100mm}|}\hline
    \textbf{ÁREA A SER OTIMIZADA}              & \textbf{COMO OTIMIZAR}                                                                                                                                                                       \\\hline
    \textbf{Quantidade de mapeadores}          & Verificar se é possível diminuir a quantidade de execuções da função \textit{Map} de modo que cada uma seja executada por mais tempo. O tempo médio recomendado na literatura é de 1 minuto. \\\hline
    \textbf{Quantidade de redutores}           & Verificar se mais de um redutor está sendo utilizado. O recomendado é que cada tarefa seja executada em média durante 5 minutos e produza 1 bloco de dados.                                  \\\hline
    \textbf{Uso de combinadores}               & Verificar se é possível utilizar algum combinador de dados de modo que a quantidade de data passada à função \textit{Shuffle} seja menor.                                                    \\\hline
    \textbf{Compressão}                        & Usualmente, o tempo de execução de um \textit{Job} é diminuído ao usar compressão de dados.                                                                                                  \\\hline
    \textbf{Ajustes na parte \textit{Shuffle}} & A parte \textit{Shuffle} do processo possui vários parâmetros de \textit{\gls{tuning}} de memória que podem ser utilizados para a melhora da performance do \textit{Job}.                    \\\hline
  \end{tabular}}
{Adaptado de \cite{HadoopBook15}}{quadro1}{}{}

\section{PARÂMETROS DO MAPREDUCE} \label{sec:parametrosmapreduce}

O \textit{\gls{tuning}} pode ser realizado através da avaliação e mudança dos parâmetros de configuração do \textit{MapReduce}. Cada parâmetro tem um objetivo específico e pode melhorar uma característica do processo. Algumas variáveis mudam configurações no \textit{Job} e algumas afetam o \textit{cluster} diretamente.

O \textit{Hadoop} foi criado para processar grandes arquivos de entradas e é otimizado para \textit{clusters} em máquinas heterogêneas, ou seja, sistemas que usam mais de um tipo de processador com o objetivo de melhorar a performance. Cada \textit{Job} segue a seguinte sequência de passos: configuração, fase \textit{shuffle/sort} e fase \textit{reduce}. O \textit{Hadoop} é responsável por configurar e gerenciar cada um desses passos \cite{ProHadoop09}.

% \textcite{HadoopBook15} explica o processo de \textit{\gls{tuning}} e explicar os parâmetros específicos para otimização da cada passo, detalhado nas seções a seguir.

% \subsection{Configuração de parâmetros}\label{ssec:configuracaooparametros}

O objetivo principal a ser atingido durante o \textit{\gls{tuning}} é possibilitar que a fase \textit{Shuffle} tenha a maior quantidade de memória disponível, ao mesmo tempo que as fases \textit{Map} e \textit{Reduce} tenham memória suficiente para funcionar propriamente. A quantidade de memória disponibilizada a cada \textit{\gls{javavirtualmachine}} é determinada pelos parâmetros \textit{mapreduce.map.memory.mb} e \textit{mapreduce.reduce.memory.mb} \cite{HadoopBook15}.

A fase \textit{Map} recebe de entrada um arquivo e o parâmetro \textit{dfs.blocksize} é responsável por determinar o tamanho do bloco em \textit{\gls{byte}s} sobre o qual esse arquivo será dividido enquanto o parâmetro \textit{dfs.replication} é responsável por determinar quantos blocos serão criados. Ainda nessa fase, os pares (CHAVE, VALOR) são particionados, e essas partições são ordenadas na fase \textit{Shuffle}. O arquivo criado para cada partição é chamado de \textit{spill}. Para cada tarefa \textit{Reduce} existe um \textit{spill}, que passará por uma ordenação do tipo \textit{\gls{mergesort}}, na segunda etapa da fase \textit{Shuffle}, chamada de \textit{Sort} \cite{ProHadoop09}.

Segundo \textcite{HadoopBook15}, a melhor performance da fase \textit{Map} pode ser obtida através da minimalização da quantidade de \textit{spills}, cujos parâmetros de controles são \textit{mapreduce.task.io.sort.factor} - número máximo de entradas para a função \textit{\gls{mergesort}} - e \textit{mapreduce.task.io.sort.mb} - tamanho do \textit{\gls{buffer}} de memória para a saída da função \textit{Map}. O último é especialmente importante e deve ser aumentado sempre que possível. Quando o \textit{\gls{buffer}} atinge a capacidade percentual determinada pelo parâmetro \textit{mapreduce.map.sort.spill.percent} ocorre um vazamento de memória e os conteúdos restantes do arquivo de saída são colocados no disco (no arquivo chamado de \textit{spill}).

Caso exista mais de uma determinada quantidade de arquivos \textit{spill} (quantidade determinada pela propriedade \textit{mapreduce.map.combine.minspills}), a função combinadora é executada novamente antes de ser criado o arquivo de saída. Uma outra otimização possível é o uso de compressores de dados nos arquivos de saída da fase \textit{Map}, processo facilmente habilitado através dos parâmetros \textit{mapreduce.map.output.compress} - valor booleano que habilita a compressão - e \textit{mapreduce.map.output.compress.codec} - classe que vai realizar a compressão \cite{HadoopBook15}.

A fase \textit{Reduce} é otimizada quando os dados intermediários são armazenados na memória, o que não acontece por padrão, visto que sem a alteração dos parâmetros toda a memória é alocada para a fase \textit{Reduce} em si. As propriedades que podem ser alteradas para atingir esse objetivo são \textit{mapreduce.reduce.merge.inmem.threshold}, \textit{mapreduce.reduce.input.buffer.percent} e \textit{mapreduce.reduce.shuffle.merge.percent}, cujos valores ótimos são 0 e 1.0, respectivamente \cite{HadoopBook15}.

Um dos parâmetros que pode ser otimizado na fase \textit{Reduce} é o \textit{mapreduce.reduce. shuffle.parallelcopies}, que determina a quantidade de tarefas que serão executadas em paralelo pelos \textit{reducers} para copiar os arquivos de saída das tarefas \textit{Map} quando estas são finalizadas e seu valor ótimo depende da quantidade de dados que já passou pela fase \textit{Shuffle} \cite{MRONLINELi14}.

Na parte do processo onde as cópias dos arquivos de saída da fase \textit{Map} são executadas, o tamanho do \textit{\gls{buffer}} é controlado pela propriedade \textit{mapreduce.reduce.shuffle.input.buffer.percent}, que especifica a proporção do \textit{\gls{heap}} que será usada para a finalidade mencionada. Quando esse \textit{\gls{buffer}} atinge um número determinado por \textit{mapreduce.reduce.shuffle.merge.percent} ou \textit{mapreduce.reduce.merge.inmem.threshold}, o restante dos arquivos é alocado no disco \cite{HadoopBook15}.

Para cada gerenciador de nodo, o número de partições do arquivo de saída disponilizados para a fase \textit{Reduce} é determinado pela propriedade \textit{mapreduce.shuffle.max.threads}. O valor padrão de 0 determina que o número máximo de tarefas é o dobro do número de processadores da máquina \cite{ProHadoop09}.

Ainda, as propriedades \textit{mapreduce.output.fileoutputformat.compress}, \textit{mapreduce.out- put.fileoutputformat.compress} e \textit{mapreduce.output.fileoutputformat.compress} determinam, respectivamente, se os arquivos de saída do \textit{Job} serão comprimidos, qual será a classe responsável pela compressão e como essa compressão ocorrerá. Por fim, o parâmetro \textit{io.file.buffer.size} determina o tamanho do \textit{\gls{buffer}} que será usado nas operações de leitura e escrita.

Os quadros a seguir resumem os parâmetros mencionados previamente, assim como outros parâmetros relevantes e os valores padrões de cada um:

\qquadro{Parâmetros de ajuste da quantidade de tarefas \textit{Map}}
{\footnotesize
  \centering
  \begin{tabular}{|p{30mm}|p{50mm}|p{35mm}|}\hline
    \textbf{PARÂMETRO}                             & \textbf{DESCRIÇÃO}                                                                                                                   & \textbf{VALOR PADRÃO} \\\hline
    \textbf{mapreduce.task. io.sort.mb}            & Tamanho em \textit{\gls{byte}s} do \textit{\gls{buffer}} de memória na ordenação na saída da função \textit{Map}.                    & 100                   \\\hline
    \textbf{mapreduce.task.io. sort.factor}        & Número máximo de arquivos para juntar simultaneamente durante a ordenação.                                                           & 10                    \\\hline
    \textbf{mapreduce.map. sort.spill.percent}     & Limite de uso do \textit{\gls{buffer}} de memória que pode ser usado antes que os dados sejam colocados em disco.                    & 0.80                  \\\hline
    \textbf{mapreduce.map. combine.minspills}      & Número mínimo de arquivos de vazamento necessário para o combinador funcionar (se um combinador for especificado).                   & 3                     \\\hline
    \textbf{mapreduce.map. output.compress}        & Define se a saída da função \textit{Map} será comprimida.                                                                            & false                 \\\hline
    \textbf{mapreduce.map. output.compress. codec} & Codificador usado na compressão.                                                                                                     & DefaultCodec          \\\hline
    \textbf{mapreduce.shuffle. max.threads}        & Número de tarefas \textit{Worker} por gerenciador de nodos. Esse parâmetro não funciona por \textit{Job} e sim por \textit{cluster}. & 0                     \\\hline
  \end{tabular}}
{Adaptado de \cite{HadoopDocs321}}{quadro2}{}{}

\qquadro{Parâmetros de ajuste da quantidade de tarefas \textit{Reduce}}
{\footnotesize
  \centering
  \begin{tabular}{|p{40mm}|p{50mm}|p{30mm}|}\hline
    \textbf{PARÂMETRO}                                      & \textbf{DESCRIÇÃO}                                                                                                                                        & \textbf{VALOR PADRÃO} \\\hline
    \textbf{mapreduce.task.io.sort. factor}                 & Número máximo de arquivos para juntar simultaneamente durante a ordenação.                                                                                & 10                    \\\hline
    \textbf{mapreduce.reduce.shuffle. parallelcopies}       & Número de tarefas usadas para copiar saída de funções \textit{Map} para funções \textit{Reduce}.                                                          & 5                     \\\hline
    \textbf{mapreduce.reduce.shuffle. maxfetchfailures}     & Número de vezes que uma tarefa \textit{Reduce} tenta obter arquivo de entrada.                                                                            & 10                    \\\hline
    \textbf{mapreduce.reduce.shuffle. input.buffer.percent} & Porcentagem de tamanho do \textit{\gls{heap}} a ser alocada para a saída da fase \textit{Map}.                                                            & 0.70                  \\\hline
    \textbf{mapreduce.reduce.shuffle. merge.percent}        & Limite porcentual da saída da fase \textit{Map} para iniciar o processo de juntar saídas.                                                                 & 0.66                  \\\hline
    \textbf{mapreduce.reduce.merge. inmem.threshold}        & Quantidade de saídas da função \textit{Map} para a saída da fase \textit{Map}.                                                                            & 1000                  \\\hline
    \textbf{mapreduce.reduce.input. buffer.percent}         & Percentual que determina o tamanho do \textit{\gls{heap}} que será utilizado para armazenar saídas da função \textit{Map} durante a fase \textit{Reduce}. & 0.0                   \\\hline
    \textbf{mapreduce.job.reduce. slowstart.completedmaps}  & Percentual de tarefas \textit{Map} que devem estar completas antes que as tarefas \textit{Reduce} sejam iniciadas.                                        & 0.05                  \\\hline
  \end{tabular}}
{Adaptado de \cite{HadoopDocs321}}{quadro3}{}{}

\qquadro{Parâmetros adicionais de configuração}
{\footnotesize
  \centering
  \begin{tabular}{|p{30mm}|p{50mm}|p{30mm}|}\hline
    \textbf{PARÂMETRO}                                          & \textbf{DESCRIÇÃO}                                                                                      & \textbf{VALOR PADRÃO} \\\hline
    \textbf{dfs.blocksize}                                      & Tamanho do bloco em \textit{\gls{byte}s} sobre o qual arquivo de entrada do \textit{Map} será dividido. & 67108864 \gls{byte}s  \\\hline
    \textbf{dfs.replication}                                    & Quantidade de replicação dos blocos.                                                                    & 3                     \\\hline
    \textbf{mapreduce.map. memory.mb}                           & Memória alocada para cada tarefa \textit{Map}.                                                          & 1024                  \\\hline
    \textbf{mapreduce.reduce. memory.mb}                        & Memória alocada para cada tarefa \textit{Reduce}.                                                       & 1024                  \\\hline
    \textbf{mapreduce.output. fileoutputformat.compress}        & Define se a saída do \textit{Job} será comprimida.                                                      & false                 \\\hline
    \textbf{mapreduce.output. fileoutputformat. compress.codec} & Codificador usado na compressão.                                                                        & DefaultCodec          \\\hline
    \textbf{mapreduce.output. fileoutputformat. compress.type}  & Define como os arquivos de saída do \textit{Job} serão comprimidos.                                     & RECORD                \\\hline
    \textbf{io.file.buffer.size}                                & Tamanho do \textit{buffer} que será usado durante operações de leitura e escrita.                       & 4096                  \\\hline
  \end{tabular}}
{Adaptado de \cite{HadoopDocs321}}{quadro4}{}{}



\section{TERASORT} \label{sec:terasort}

A técnica de \textit{benchmarking} consiste na execução, por diversas vezes, de um programa (no caso do \textit{MapReduce}, de um \textit{Job}), a fim de testar se os resultados obtidos são os esperados. Esse processo é eficiente pois é possível comparar os diversos resultados obtidos e obter avaliações de performance \cite{HadoopBook15}.

O \textit{Hadoop} possui várias métricas de \textit{benchmarks} imbutidas que podem ser utilizadas para melhorar o funcionamento do \textit{Job}, cada uma delas com o objetivo de monitorar um fator diferente, como por exemplo, TestDFSIO  - responsável por testar a performance dos dispositivos de entrada e saída - e MRBench/NNBench - testam em conjunto vários \textit{Jobs} pequenos múltiplas vezes \cite{HadoopBook15}.

Nesse trabalho será utilizado a ferramenta \textit{TeraSort}, que ordena o arquivo de entrada completamente. Para \textcite{HadoopBook15}, ela é extremamente eficaz no \textit{benchmarking} do \textit{HDFS} e \textit{MapReduce} em conjunto, já que todo os dados de entrada são processados. Essa ferramenta funciona em três etapas:

\begin{itemize}
  \item \textit{TeraGen} executa um \textit{Job} só de funções \textit{Map} que cria um conjunto de dados binários aleatórios. A execução desse comando é exemplificada no \autoref{code:codigo3}.
  
  \begin{lstlisting}[caption={Exemplo de execução do \textit{TeraGen} adaptado de \cite{HadoopBook15}}, label=code:codigo3]
  hadoop jar hadoop-mapreduce-examples-*.jar \
  teragen <numero de linhas de 100 bytes cada> <diretorio de saida>
  \end{lstlisting}
  
  \item \textit{TeraSort} executa a ordenação dos dados. É nesse passo que é avaliada a performance do \textit{MapReduce}, pois é aqui que as operações do paradigma em si são executadas. A execução desse comando é exemplificada no \autoref{code:codigo4}.
  \begin{lstlisting}[caption={Exemplo de execução do \textit{TeraSort} adaptado de \cite{HadoopBook15}}, label=code:codigo4]
  hadoop jar hadoop-mapreduce-examples-*.jar \
  terasort <diretorio de entrada> <diretorio de saida>
  \end{lstlisting}  

  \item \textit{TeraValidate} executa checagens nos dados ordenados resultantes da fase anterior para verificar se a ordenação foi feita corretamente. A execução desse comando é exemplificada no \autoref{code:codigo5}.
  \begin{lstlisting}[caption={Exemplo de execução do \textit{TeraValidate} adaptado de \cite{HadoopBook15}}, label=code:codigo5]
  hadoop jar hadoop-mapreduce-examples-*.jar \
  teravalidate <arquivo de entrada (diretorio de saida do terasort)> <diretorio de saida>
  \end{lstlisting}

\end{itemize}

Os parâmetros de configuração do \textit{tuning} podem ser usados no comando que executa o \textit{TeraSort}, processo exemplificado na \autoref{fig:fig5}.

\figura{Execução do \textit{TeraSort} com parâmetros de \textit{tuning}}{.900}{fig/fig5.png}{A autora (2022)}{fig5}{}{}
\section{AMBIENTE EXPERIMENTAL} \label{sec:ambienteexperimental}

O ambiente no qual serão realizadas as execuções e testes tem as seguintes configurações: LENOVO Ideapad 310 com processador Intel(R) Core(TM) i5-6200U, 2.30GHz de velocidade de processamento,  memória RAM de 8GB, armazenamento de disco de 1TB e SSD Kingston A400 de 480, com leitura de 500 MB/s e gravação de 450 MB/s. 

No entanto, a imagem Docker é executada no Windows WSL2, Substema do Windows para Linux, que possibilitam aos programadores executar um ambiente Linux mesmo usando um sistema Windows \cite{MicrosoftWSL22}. Nesse ambiente virtual, está sendo executado Linux na distribuição Ubuntu 20.04, Docker na versão 20.10.12 e Hadoop na versão 3.2.1.

Usando a imagem do \textit{Hadoop} para Docker disponibilizada pelo Big Data Europe \cite{BigDataHadoopGithub}, foi configurado um contêiner Docker com um \textit{cluster Hadoop} com 3 \textit{datanodes (workers)}, um \textit{namenode HDFS (master)}, um gerenciador de recursos YARN, um servidor com histórico de operações e um gerenciador de nodos.

\section{RESULTADOS} \label{sec:resultados}

Para que o objetivo de melhora de performance através do \textit{\gls{tuning}} dos parâmetros fosse atingido, foram feitos testes repetidos utilizando o \textit{Terasort} com variação dos valores padrões de cada uma das propriedades relevantes mencionadas previamente de acordo com sugestões de \textcite{HadoopBook15} e \textcite{ProHadoop09}, assim como a análise de cada uma das execuções da ferramenta de \textit{\gls{benchmark}ing} para que fossem decididas quais mudanças teriam mais impacto no resultado. As execuções do \textit{TeraSort} foram realizadas após a geração de dados aleatórios de entrada pelo \textit{TeraGen} de um arquivo de 10GB.

Os primeiros ajustes realizados foram nas propriedades relacionadas à quantidade de memória disponível, isto é, na memória disponibilizada aos \textit{\gls{buffer}s} utilizados nas operações de leitura e escrita e na saída da função \textit{Map}, assim como às funções \textit{Map} e \textit{Reduce}, o que se provou de grande importância porque diminiui consideralmente o tempo de execução do programa. Isso ocorre devido ao fato de que, ao disponibilizar mais memória aos programas, menos dados são copiados para o disco, economizando tempo.

Depois, foram feitas alterações nas propriedades referentes a compressão de arquivos, habilitando-se a compressão na saída da fase \textit{Map} e do \textit{Job} utilizando a classe \textit{org.apache.hadoop.io.compress.Lz4Codec} para realizar essa operação. A compressão é relevante porque permite que os recursos computacionais não fiquem parados esperando operações de entrada e saída em disco finalizarem por estarem trabalhando com arquivos muito grandes.

Os parâmetros relativos aos blocos sobre os quais os arquivos de entrada da função \textit{Map} são separados também foram modificados, tendo seu tamanho aumentado e sua taxa de replicação diminuída. Esses dois fatores permitem que os blocos sejam de maior tamanho, mas não sejam duplicados em nodos diferentes, resultando em menos operações de leitura e escrita e menos uso de rede.

Outras propriedades da tarefa \textit{Map} que foram alteradas são as que determinam a quantidade de arquivos para juntar simultaneamente e a da taxa limite do \textit{\gls{buffer}} para que os valores deste sejem transferidos para o disco.

Em relação à fase \textit{Shuffle}, o único parâmetro modificado foi o \textit{mapreduce.shuffle.max. threads}, determinando-se que apenas uma tarefa \textit{Worker} fosse usada pelo gerenciador de nodos.

Por fim, na fase \textit{Reduce}, as propriedades que tiveram efeito relevante na performance foram as que determinam a quantidade de transferências paralelas executadas pela função \textit{Reduce} durante a fase de cópia dos arquivos da fase \textit{Shuffle}, o limite de arquivos para o processo de junção antes de acontecem transferência para o disco, a quantidade de memória a ser alocada para armazenar arquivos de saída da fase \textit{Map} durante a fase \textit{Shuffle} e a porcentagem de tarefas \textit{Map} que deveriam estar finalizar antes de ser iniciado o processo de \textit{Reduce}.

Os parâmetros e seus valores que foram alterados por afetarem a performance estão descritos no \autoref{qua:quadro5}.

\qquadro{Parâmetros ajustados durante o \textit{tuning}}
{\footnotesize
  \centering
  \begin{tabular}{|p{80mm}|p{25mm}|}\hline
    \textbf{PARÂMETRO}                                        & \textbf{VALOR FINAL} \\\hline
    \textbf{mapreduce.map.output.compress}                    & true                 \\\hline
    \textbf{mapreduce.map.output.compress.codec}              & Lz4Codec             \\\hline
    \textbf{dfs.blocksize}                                    & 335544320            \\\hline
    \textbf{dfs.replication}                                  & 1                    \\\hline
    \textbf{mapreduce.output.fileoutputformat.compress}       & true                 \\\hline
    \textbf{mapreduce.output.fileoutputformat.compress.codec} & Lz4Codec             \\\hline
    \textbf{mapreduce.output.fileoutputformat.compress.type}  & BLOCK                \\\hline
    \textbf{mapreduce.map.memory.mb}                          & 2048                 \\\hline
    \textbf{io.file.buffer.size}                              & 131072               \\\hline
    \textbf{mapreduce.task.io.sort.mb}                        & Lz4Codec             \\\hline
    \textbf{mapreduce.task.io.sort.factor}                    & 256                  \\\hline
    \textbf{mapreduce.map.sort.spill.percent}                 & 400                  \\\hline
    \textbf{mapreduce.shuffle.max.threads}                    & 1.0                  \\\hline
    \textbf{mapreduce.reduce.shuffle.parallelcopies}          & 20                   \\\hline
    \textbf{mapreduce.reduce.merge.inmem.threshold}           & 2000                 \\\hline
    \textbf{mapreduce.reduce.input.buffer.percent}            & 0.8                  \\\hline
    \textbf{mapreduce.job.reduce.slowstart.completedmaps}     & 0.7                  \\\hline
  \end{tabular}}
{A autora(2022)}{quadro5}{}{}

Alguns parâmetros mencionados anteriormente - resumidos no \autoref{qua:quadro6} não tiveram efeito na performance no programa, seja pelo tamanho do arquivo de dados inicial pelas configurações do ambiente experimental. Por causa disso, seus valores padrões foram mantidos nas execuções.

\qquadro{Parâmetros não ajustados durante o \textit{tuning}}
{\footnotesize
  \centering
  \begin{tabular}{|p{75mm}|p{25mm}|}\hline
    \textbf{PARÂMETRO}                                     & \textbf{VALOR FINAL} \\\hline
    \textbf{mapreduce.map.combine.minspills}               & 3                    \\\hline
    \textbf{mapreduce.reduce.shuffle.maxfetchfailures}     & 10                   \\\hline
    \textbf{mapreduce.reduce.shuffle.input.buffer.percent} & 0.7                  \\\hline
    \textbf{mapreduce.reduce.shuffle.merge.percent}        & 0.66                 \\\hline
  \end{tabular}}
{A autora(2022)}{quadro6}{}{}

Os resultados relevantes da aplicação do \textit{\gls{tuning}} são os tempos obtidos na execução do \textit{TeraSort}, os quais podem ser obtidos pela sua saída que mostra, em milisegundos, o tempo utilizado pelas fases \textit{Map} e \textit{Reduce} do processo. Segundo \textcite{Fleming86}, no caso de \textit{\gls{benchmark}ing} de métricas de tempo, é apropriado usar uma média aritmética padrão para avaliar os resultados. Além disso, apresentar os valores mínimos e máximos obtidos na execuções de modo que uma visão geral da performance possa ser exemplificada.

Para geração dessas métricas, o \textit{TeraSort} foi executado 10 vezes com os parâmetros com valores padrão e 10 vezes com os parâmetros com valores alterados.


\figura{Resultados do \textit{tuning}}{.900}{fig/fig6.png}{A autora (2022)}{fig6}{}{}
\newpage
Dessa forma, a \autoref{fig:fig6} acima ilustra os resultados obtidos nesse experimentos antes e depois do processo de aplicação do \textit{\gls{tuning}}. Como é possível ver, a mudança dos parâmetros teve um grande impacto da performance do \textit{Hadoop MapReduce}, diminuindo a execução das suas principais funções em quase 50\%. 
