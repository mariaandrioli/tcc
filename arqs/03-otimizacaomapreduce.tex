\chapter{OTIMIZAÇÃO DO MAPREDUCE} \label{cha:otimizacaomapreduce}

Como visto anteriormente, o \textit{MapReduce} é um processo com diversas etapas, e portanto, há muitas permutações possíveis das suas configurações que permitem que sua performance seja melhorada. A melhora da performance através da modificação dos parâmetros de configuração é chamado de \textit{tuning}. Para \textcite{HadoopBook15}, existem algum fatores do \textit{Job} a serem considerados - exibidos no \autoref{qua:quadro1} - com objetivo de obter aumento da performance.

\qquadro{Fatores para \textit{Tuning} do \textit{Job MapReduce}}
{\footnotesize
  \centering
  \begin{tabular}{|p{50mm}|p{100mm}|}\hline
    \textbf{ÁREA A SER OTIMIZADA}     & \textbf{COMO OTIMIZAR}                                                                                                                                                                       \\\hline
    \textbf{Quantidade de mapeadores} & Verificar se é possível diminuir a quantidade de execuções da função \textit{Map} de modo que cada uma seja executada por mais tempo. O tempo médio recomendado na literatura é de 1 minuto. \\\hline
    \textbf{Quantidade de redutores}  & Verificar se mais de um redutor está sendo utilizado. O recomendado é que cada tarefa seja executada em média durante 5 minutos e produza 1 bloco de dados.                                  \\\hline
    \textbf{Uso de combinadores}      & Verificar se é possível utilizar algum combinador de dados de modo que a quantidade de data passada à função \textit{Shuffle} seja menor.                                                    \\\hline
    \textbf{Compressão}               & Usualmente, o tempo de execução de um \textit{Job} é diminuídp ao usar compressão de dados.                                                                                                  \\\hline
    \textbf{Ajustes na parte Shuffle} & A parte \textit{Shuffle} do processo possui vários parâmetros de \textit{tuning} de memória que podem ser utilizados para a melhora da performance do \textit{Job}.                          \\\hline
  \end{tabular}}
{Adaptado de \cite{HadoopBook15}}{quadro1}{}{}

\section{PARÂMETROS DO \textit{MAPREDUCE}} \label{sec:parametrosmapreduce}

Como mencionado, o \textit{tuning} pode ser realizado através da avaliação e mudança dos parâmetros de configuração do \textit{MapReduce}. Cada parâmetro tem um objetivo específico e pode melhorar uma característica do processo. Algumas variáveis mudam configurações no \textit{Job} e algumas afetam o \textit{cluster} diretamente.

\textit{Hadoop} foi feito para processar grandes arquivos de entradas e é otimizado para \textit{clusters} em máquinas heterogêneas, ou seja, sistemas que usam mais de um tipo de processador com o objetivo de melhorar a performance. Cada \textit{Job} segue a seguinte sequência de passos: configuração, fase \textit{shuffle/sort} e fase \textit{reduce}. O \textit{Hadoop} é responsável por configurar e gerenciar cada um desses passos \cite{ProHadoop09}.

Uma das seções do capítulo 7 de \textcite{HadoopBook15} é dedicado ao processo de \textit{tuning} e explicar os parâmetros específicos para otimização da cada passo, que são explicados a seguir.

\subsection{Configuração de parâmetros}\label{ssec:configuracaooparametros}

Com o propósito de atingir os objetivos demonstrados no \autoref{qua:quadro1}, \textcite{HadoopBook15} sugere configurar os seguintes parâmetros, descritos no \autoref{qua:quadro2} e \autoref{qua:quadro3}:

\qquadro{Parâmetros de ajuste da quantidade de tarefas \textit{Map}}
{\footnotesize
  \centering
  \begin{tabular}{|p{30mm}|p{50mm}|p{35mm}|}\hline
    \textbf{PARÂMETRO}                         & \textbf{DESCRIÇÃO}                                                                                                                   & \textbf{VALOR PADRÃO} \\\hline
    \textbf{mapreduce.task. io.sort.mb}        & Tamanho em \textit{bytes} usado no \textit{buffer} de memória na ordenação na saída da função \textit{Map}.                          & 100                   \\\hline
    \textbf{mapreduce.map. sort.spill.percent} & Limite de uso do \textit{buffer} de memória e de início do processo de vazamento em disco.                                           & 0.80                  \\\hline
    \textbf{mapreduce.task. io.sort.factor}    & Número máximo de entradas para a função de junção na ordenação. \textit{bytes}                                                       & 10                    \\\hline
    \textbf{mapreduce.map. combine.minspills}  & Número mínimo de arquivos de vazamento necessário para o combinador funcionar (se um combinador for especificado)                    & 3                     \\\hline
    \textbf{mapreduce.map. output.compress}    & Define se a saída da função \textit{Map} será comprimida.                                                                            & false                 \\\hline
    \textbf{mapreduce.map. output.compress}    & Codificador usado na compressão.                                                                                                     & DefaultCodec (classe) \\\hline
    \textbf{mapreduce.shuffle. max.threads}    & Número de tarefas \textit{Worker} por gerenciador de nodos. Esse parâmetro não funciona por \textit{Job} e sim por \textit{cluster}. & 0                     \\\hline
  \end{tabular}}
{Adaptado de \cite{HadoopBook15}}{quadro2}{}{}

\qquadro{Parâmetros de ajuste da quantidade de tarefas \textit{Map}}
{\footnotesize
  \centering
  \begin{tabular}{|p{40mm}|p{50mm}|p{30mm}|}\hline
    \textbf{PARÂMETRO}                                       & \textbf{DESCRIÇÃO}                                                                                                                                                                      & \textbf{VALOR PADRÃO} \\\hline
    \textbf{mapreduce.reduce.shuffle. parallelcopies}        & Número de tarefas usadas para copiar saída de funções \textit{Map} para funções \textit{Reduce}.                                                                                        & 5                     \\\hline
    \textbf{mapreduce.reduce.shuffle. maxfetchfailures}      & Número de vezes que uma tarefa \textit{Reduce} tenta obter arquivo de entrada.                                                                                                          & 10                    \\\hline
    \textbf{mapreduce.task.io.sort. factor}                  & Número máximo de arquivos para juntar simultaneamente durante a ordenação.                                                                                                              & 10                    \\\hline
    \textbf{mapreduce.reduce.shuffle. input.buffer.percent} & Porcentagem de tamanho do \textit{heap} a ser alocada para a saída da fase \textit{Map}.                                                                                                & 0.70                  \\\hline
    \textbf{mapreduce.reduce.shuffle. merge.percent}         & Limite porcentual da saída da fase \textit{Map} para iniciar o processo de juntar saídas.                                                                                               & 0.66                  \\\hline
    \textbf{mapreduce.reduce.merge. inmem.threshold}         & Quantidade de saídas da função \textit{Map} para a saída da fase \textit{Map}. Se for igual ou menor a zero, esse fator é definido apenas pelo mapreduce.reduce. shuffle.merge.percent. & 1000                  \\\hline
    \textbf{mapreduce.reduce.input. buffer.percent}          & Percentual que determinar o tamanho do \textit{heap} que será utilizado para armazenar saídas da função \textit{Map} durante a fase \textit{Reduce}.                                    & 0.0                   \\\hline
  \end{tabular}}
{Adaptado de \cite{HadoopBook15}}{quadro3}{}{}

% \subsection{Shuffle e sort}\label{ssec:shufflesort}
% O JobTracker tem um número determiado de lugares para a execução das tarefas Map. Cada divisão feita no split é alocada um desses lugares para ser executada. Manda tarefas para um lugar local da máquina melhora a performance do dispositivo de entrada e saída. Se existem lugares sobrando e o map speculative execution está habilitado, várias instância da uma tarefa map podem ser agendadas para serem executadas. Se ele valor nao estiver habilitado, apenas uma instância de tarefa map será executada por vez.
\section{TERASORT} \label{sec:terasort}

A técnica de \textit{benchmarking} consiste na execução, por diversas vezes, de um programa (no caso do \textit{MapReduce}, de um \textit{Job}), a fim de testar se os resultados obtidos são os esperados. Esse processo é eficiente pois é possível comparar os diversos resultados obtidos e obter avaliações de performance \cite{HadoopBook15}.

O \textit{Hadoop} possui várias métricas de \textit{benchmarks} imbutidas que podem ser utilizadas para melhorar o funcionamento do \textit{Job}, cada uma delas com o objetivo de monitorar um fator diferente, como por exemplo, TestDFSIO  - responsável por testar a performance dos dispositivos de entrada e saída - e MRBench/NNBench - testam em conjunto vários \textit{Jobs} pequenos múltiplas vezes \cite{HadoopBook15}.

Nesse trabalho será utilizado a ferramenta \textit{TeraSort}, que ordena o arquivo de entrada completamente. Para \textcite{HadoopBook15}, ela é extremamente eficaz no \textit{benchmarking} do \textit{HDFS} e \textit{MapReduce} em conjunto, já que todo os dados de entrada são processados. Essa ferramenta funciona em três etapas:

\begin{itemize}
  \item \textit{TeraGen} executa um \textit{Job} só de funções \textit{Map} que cria um conjunto de dados binários aleatórios. A execução desse comando é exemplificada no \autoref{code:codigo3}.
  
  \begin{lstlisting}[caption={Exemplo de execução do \textit{TeraGen} adaptado de \cite{HadoopBook15}}, label=code:codigo3]
  hadoop jar hadoop-mapreduce-examples-*.jar \
  teragen <numero de linhas de 100 bytes cada> <diretorio de saida>
  \end{lstlisting}
  
  \item \textit{TeraSort} executa a ordenação dos dados. É nesse passo que é avaliada a performance do \textit{MapReduce}, pois é aqui que as operações do paradigma em si são executadas. A execução desse comando é exemplificada no \autoref{code:codigo4}.
  \begin{lstlisting}[caption={Exemplo de execução do \textit{TeraSort} adaptado de \cite{HadoopBook15}}, label=code:codigo4]
  hadoop jar hadoop-mapreduce-examples-*.jar \
  terasort <diretorio de entrada> <diretorio de saida>
  \end{lstlisting}  

  \item \textit{TeraValidate} executa checagens nos dados ordenados resultantes da fase anterior para verificar se a ordenação foi feita corretamente. A execução desse comando é exemplificada no \autoref{code:codigo5}.
  \begin{lstlisting}[caption={Exemplo de execução do \textit{TeraValidate} adaptado de \cite{HadoopBook15}}, label=code:codigo5]
  hadoop jar hadoop-mapreduce-examples-*.jar \
  teravalidate <arquivo de entrada (diretorio de saida do terasort)> <diretorio de saida>
  \end{lstlisting}

\end{itemize}

Os parâmetros de configuração do \textit{tuning} podem ser usados no comando que executa o \textit{TeraSort}, processo exemplificado na \autoref{fig:fig5}.

\figura{Execução do \textit{TeraSort} com parâmetros de \textit{tuning}}{.900}{fig/fig5.png}{A autora (2022)}{fig5}{}{}

 
% ---
% Capitulo que faz uso de elementos do glossario
% ---
\chapter{Orientações a respeito de glossários}
 
% ---
\section{Usar o glossário no texto}
% ---
\begin{citacao}
Esta frase usa a palavra \gls{latex} \gls{teste} \gls{aloha}
\end{citacao}


\section{AMBIENTE EXPERIMENTAL} \label{sec:ambienteexperimental}

O ambiente no qual serão realizadas as execuções e testes tem as seguintes configurações: LENOVO Ideapad 310 com processador Intel(R) Core(TM) i5-6200U, 2.30GHz de velocidade de processamento,  memória RAM de 8GB, armazenamento de disco de 1TB e SSD Kingston A400 de 480, com leitura de 500 MB/s e gravação de 450 MB/s. 

No entanto, a imagem Docker é executada no Windows WSL2, Substema do Windows para Linux, que possibilitam aos programadores executar um ambiente Linux mesmo usando um sistema Windows \cite{MicrosoftWSL22}. Nesse ambiente virtual, está sendo executado Linux na distribuição Ubuntu 20.04, Docker na versão 20.10.12 e Hadoop na versão 3.2.1.

Usando a imagem do \textit{Hadoop} para Docker disponibilizada pelo Big Data Europe \cite{BigDataHadoopGithub}, foi configurado um contêiner Docker com um \textit{cluster Hadoop} com 3 \textit{datanodes (workers)}, um \textit{namenode HDFS (master)}, um gerenciador de recursos YARN, um servidor com histórico de operações e um gerenciador de nodos.
