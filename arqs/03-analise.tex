\chapter{ANÁLISE DOS PARÂMETROS DE CONFIGURAÇÃO} \label{cha:analiseparamconfig}

Como visto anteriormente, o \textit{MapReduce} é um processo com diversas etapas, e portanto, há muitas permutações possíveis das suas configurações que permitem que sua performance seja melhorada. A melhora da performance através da modificação dos parâmetros de configuração é chamado de \textit{tuning}. Para \textcite{HadoopBook15}, existem algum fatores do \textit{Job} a serem considerados - exibidos no \autoref{qua:quadro1} - do \textit{Job} com o objetivo de obter aumento da performance.

\qquadro{Fatores para \textit{Tuning} do \textit{Job MapReduce}}
{\footnotesize
  \centering
  \begin{tabular}{|p{50mm}|p{100mm}|}\hline
    \textbf{ÁREA A SER OTIMIZADA} & \textbf{COMO OTIMIZAR} \\\hline
    \textbf{Quantidade de mapeadores} & Verificar se é possível diminuir a quantidade de execuções da função \textit{Map} de modo que cada uma seja executada por mais tempo. O tempo médio recomendado na literatura é de 1 minuto.  \\\hline
    \textbf{Quantidade de redutores} & Verificar se mais de um redutor está sendo utilizado. O recomendado é que cada tarefa seja executada em média durante 5 minutos e produza 1 bloco de dados.  \\\hline
    \textbf{Uso de combinadores} & Verificar se é possível utilizar algum combinador de dados de modo que a quantidade de data passada à função \textit{Shuffle} seja menor.  \\\hline
    \textbf{Compressão} & Usualmente, o tempo de execução de um \textit{Job} é diminuídp ao usar compressão de dados. \\\hline
    \textbf{Ajustes na parte Shuffle} & A parte \textit{Shuffle} do processo possui vários parâmetros de \textit{tuning} de memória que podem ser utilizados para a melhora da performance do \textit{Job}. \\\hline
\end{tabular}}
{Adaptado de \cite{HadoopBook15}}{quadro1}{}{}

[TODO: quais parametros serao configurados e porque - ProHadoop, HadoopBook]
[TODO: metodologia: explicar teragen, terasort etc]
